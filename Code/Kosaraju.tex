\subsubsection{Kosaraju}
\begin{verbatim}
#include<cstdio>
#include<utility>
#include<cstring>
#include<iostream>
using namespace std;

#define MP make_pair

const int maxn=100001;
const int kMaxM=1000001;

int n,m,tot,top,total;
bool vis[maxn];
int g[2][maxn],pr[maxn],part[maxn],adj[kMaxM*2],next[kMaxM*2];
//模拟栈,纪录dfs时候的局部变量
pair<int,int> stk[maxn];

void init()
{
  int x,y;
  scanf("%d%d",&n,&m);
  total=0;
  memset(g,0,sizeof(g));
  for (int i=1; i<=m; i++)
  {
    scanf("%d%d",&x,&y);
    ++total;
    adj[total]=y;
    next[total]=g[0][x];
    g[0][x]=total;

    ++total;
    adj[total]=x;
    next[total]=g[1][y];
    g[1][y]=total;
  }
}

void dfs1()
{
  while (top>0)
  {
    int now=stk[top].first;
    int &p=stk[top].second;
    if (p!=0)
    {
      int y=adj[p];
      p=next[p];
      if (!vis[y])
      {
        vis[y]=true;
        stk[++top]=MP(y,g[0][y]);
      }
    } else
    {
      ++tot;
      pr[tot]=now;
      --top;
    }
  }
}

void dfs2()
{
  while (top>0)
  {
    int now=stk[top].first;
    int &p=stk[top].second;
    if (p!=0)
    {
      int y=adj[p];
      p=next[p];
      if (!vis[y])
      {
        vis[y]=true;
        stk[++top]=MP(y,g[1][y]);
      }
    } else
    {
      part[now]=tot;
      --top;
    }
  }
}

void solve()
{
  memset(vis,0,sizeof(vis));
  tot=0;
  for (int i=1; i<=n; i++)
    if (!vis[i])
    {
      vis[i]=true;
      top=0;
      stk[++top]=MP(i,g[0][i]);
      dfs1();
    }
  memset(vis,0,sizeof(vis));
  tot=0;
  for (int i=n; i>=1; i--)
    if (!vis[pr[i]])
    {
      vis[pr[i]]=true;
      top=0; ++tot;
      stk[++top]=MP(pr[i],g[1][pr[i]]);
      dfs2();
    }
  for (int i=1; i<=n; i++) cout<<part[i]<<endl;
}

int main()
{
  init();
  solve();
  return 0;
}
\end{verbatim} 
