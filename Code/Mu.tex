\subsection{莫比乌斯函数}
当$n=1$时$\mu(n)=1$,当$n=p_1p_2\cdots p_k$时,$\mu(n)=(-1)^k$,其余情况$\mu(n)=0$.

\begin{verbatim}
void calcmu(int n){
    memset(check,0,sizeof(check));
    mu[1] = 1; total = 0;
    for (int i = 2; i <= n; i++){
        if (!check[i]){
            prime[total++] = i;
            mu[i] = -1;
        }
        for (int j = 0; j < total; j++){
            if (i * prime[j] > n) break;
            check[i * prime[j]] = true;
            if (i % prime[j] == 0){
                mu[i * prime[j]] = 0;
                break;
            } else{
                mu[i * prime[j]] = -mu[i];
            }
        }
    }
    for (int i = 1; i <= n; i++) printf("%d %d\n",i,mu[i]);
}
\end{verbatim}
