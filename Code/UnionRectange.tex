\subsection{矩形面积并}

\begin{verbatim}
/*
 * 求n个矩形的面积并
 * 先离散化x轴坐标和y轴坐标
 * 用平行x轴的扫描线扫描,把x轴的线段插入到当前的线段树里面
 * 线段树记录x轴被覆盖的长度
 * 假设离散化后x坐标有m个,那么线段树记录的下标是[1..m-1]
 * 若下标从0开始计数,那么区间[l,r]的长度为x[r]-x[l-1]
*/

#include <cstdio>
#include <cstring>
#include <iostream>
#include <algorithm>
using namespace std;

const int kMaxN=100001;

struct segment{
    int x1,x2,y,flag;
    segment(){}
    segment(int x1, int x2, int y, int flag):x1(x1),x2(x2),y(y),flag(flag){}
};

struct data{
    int cover,len;
};

data tree[kMaxN*8];
segment seg[kMaxN*2];
int n,tot,totx,toty;
int vx[kMaxN*2],vy[kMaxN*2];

bool cmp(const segment &a, const segment &b){
    return a.y<b.y;
}

void init(){
    int x1,y1,x2,y2;
    tot=totx=toty=0;
    cin>>n;
    for (int i=0; i<n; i++){
        cin>>x1>>y1>>x2>>y2;
        vx[totx++]=x1,vx[totx++]=x2;
        vy[toty++]=y1,vy[toty++]=y2;
        seg[tot++]=segment(x1,x2,y1,1);
        seg[tot++]=segment(x1,x2,y2,-1);
    }
    sort(vx,vx+totx);
    sort(vy,vy+toty);
    totx=unique(vx,vx+totx)-vx;
    toty=unique(vy,vy+toty)-vy;
    sort(seg,seg+tot,cmp);
}

void update(int l, int r, int state){
    if (!tree[state].cover){
        tree[state].len=tree[state*2].len+tree[state*2+1].len;
    } else tree[state].len=vx[r]-vx[l-1];
}

void insert(int l, int r, int state, int L, int R, int flag){
    if (L>r || l>R) return;
    if (L<=l && r<=R){
        tree[state].cover+=flag;
        update(l,r,state);
        return;
    }

    int mid=(l+r)/2;
    insert(l,mid,state*2,L,R,flag);
    insert(mid+1,r,state*2+1,L,R,flag);
    update(l,r,state);
}

void solve(){
    memset(tree,0,sizeof(tree));
    long long area=0,t=0;
    for (int i=0; i<toty-1; i++){
        while (t<tot && seg[t].y==vy[i]){
            int L=lower_bound(vx,vx+totx,seg[t].x1)-vx+1,R=lower_bound(vx,vx+totx,seg[t].x2)-vx;
            insert(1,totx-1,1,L,R,seg[t].flag);
            ++t;
        }
        area+=(long long)(tree[1].len)*(vy[i+1]-vy[i]);
    }
    cout<<area<<endl;
}

int main(){
    int ca;
    cin>>ca;
    for (int i=0; i<ca; i++){
        init();
        solve();
    }
    return 0;
}

\end{verbatim}
