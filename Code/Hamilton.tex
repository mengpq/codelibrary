\subsection{哈密尔顿回路}

求一个图的哈密尔顿回路是一个NP问题,只能用搜索解决。当定点数$n$比较小的时候,可以用状态压缩Dp解决。当$n$比较大的时候,只能用
搜索解决。但是,关于哈密尔顿回路,有一个性质:\par
Ore性质:对所有不邻接的不同顶点对$x$和$y$,有$$deg(x)+deg(y)\geq n$$
那么这个图一定存在哈密尔顿回路,且可以用一下方法求回路,时间复杂度接近$O(n^2)$ \par

~\\

1)从任意一个顶点开始,在它的任意一端邻接一个顶点,构造一条越来越长的路径,直到不能再加长为止。设路径为$$\gamma:y_1-y_2-\cdots-y_m$$

~\\

2)检查$y_1$和$y_m$是否邻接。\par
.\quad \quad a)如果$y_1$和$y_m$不邻接,则转到3,否则,$y_1$和$y_m$是邻接的,转到b。\par
.\quad \quad b)如果$m=n$,则停止构造并输出哈密尔顿回路$y_1-y_2-\cdots-y_m-y_1$,否则,转到c。\par
.\quad \quad
c)找出一个不在$\gamma$上的顶点$z$和在$\gamma$上的顶点$y_k$,满足$z$和$y_k$是邻接的,将$\gamma$用下面的长度为$m+1$的路径来替代
$$z-y_k-\cdots-y_m-y_1-\cdots-y_{k-1}$$
.\quad \quad 转到2)

~\\

3)找出一个顶点$y_k(1<k<m)$,满足$y_1$和$y_k$是邻接的,且$y_{k-1}$和$y_{m}$也是邻接的,将$\gamma$用下面的路径来替代
$$y_1-\cdots-y_{k-1}-y_{m}-\cdots-y_k$$
.\quad \quad 转到2)
