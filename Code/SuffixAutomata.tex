\subsubsection{后缀自动机} 
~\\
1、后缀自动机的的节点分为接受态和非接受态两种,接受态表示这个节点可以接收当前字符串的后缀,即从出发点走到接受态的节点的路径(字符串)是当前字符串的后缀从。从结束态(最后一个字符)沿着parent一直走到初始状态,所经过的节点就是接受态的节点。\par
~\\
2、构造字符串S的后缀自动机,从起始点出发,到达任意一个接受态的路径是字符串S的后缀,从起始点到一个接受态a,最多不会超过n条路径(表示最多有n个后缀),对于两个不同的接受态a和b,从起始点出发到a,b的路径集合要么是包含关系,要么是不相交。\par
~\\
3、从起始点出发,到达任意一个顶点a的路径p是字符串S的一个子串,后缀自动机保证了从起始点出发到达任意点的路径都是不同的。假设从起始点出发,到达一个节点$a_i$的路径数$n_i$,那么字符串S的所有不同的子串的个数为$\sum_{a_i}n_i$。\par
~\\
4、假设从出发点到一个节点$a$的所有路径为$\{p_i\}$,那么对于任意$len(p_i) < len(p_j)$,$p_i$是$p_j$的后缀。\par
~\\
5、构造完后缀自动机之后,要解决问题一般可以dfs遍历自动机,或者用桶排序根据自动机的每个节点的maxL构造出拓扑图,然后在拓扑图上求解。\par
~\\

题目:\par
\begin{itemize}
    \item 1.给出字符串$S$以及$q$个询问,每个询问给出一个字符串$sub$,问字符串$sub$在$S$出现的次数。
        \begin{itemize}
            \item 从起始点出发,按照$sub$的字母顺序走到状态$a$,假设状态$a$有$m$条不同的路径能走到某个接受态,那么$sub$在$S$中出现的次数为$m$。
        \end{itemize}
    \item 2.给出字符串$S_1$,$S_2$,求这两个串的最长公共子串。
        \begin{itemize}
            \item 首先用$S_1$构造后缀自动机,然后按照$S_2$的字母顺序在自动机上遍历。
            \item 假设当前自动机上的节点为$a$,对于$S_2[i]$,如果$a$有$S_2[i]$这个儿子,那么当前的长度maxL + 1,将$a$转移到$a->ch[S_2[i]]$。如果$a$没有$S_2[i]$这个儿子,那么沿着$a$的parent往走,走到$a$为空或者$a$存在一个$S_2[i]$的儿子节点为止,当$a$存在一个$S_2[i]$的儿子,将maxL设为$a->ch[S_2[i]]->maxL$,否则$maxL = 0$。
            \item 每走一步都尝试更新最长公共字串的答案。
        \end{itemize}
    \item 3.求多个字符串$S_1, S_2, \cdots, S_n$的最长公共子串。
        \begin{itemize}
            \item 首先选一个字符串$S_1$构造后缀自动机。
            \item 对于其他$S_i$,按照字母顺序在自动机上遍历,走到自动机上的一个状态$a$,更新这个状态的$maxL$得到最大值,设为$maxL_i[a]$。
            \item 因为从起始点出发,到达同一个节点$a$的路径两两之间的有后缀关系,所以对于某个某个状态$a$,$S_2,\cdots,S_n$走到这个状态的$maxL$的最小值就是走到这个状态的所有串的最长公共子串。所以$n$个串的最长公共字串的值就是$max(min(maxL_i[a])), (2\leq i\leq n)$。
        \end{itemize}
\end{itemize}

~\\
以下程序为给出$n$个串,求出他们的最长公共子串的长度。

\begin{verbatim}
#include <stdio.h>
#include <string.h>

const int kMaxN = 100001;

struct SAMnode{
    int maxL;
    SAMnode *ch[26], *parent;
};

struct SAM{
    int cnt;
    SAMnode *start, *end;
    SAMnode pool[kMaxN * 2];

    void init(){
        cnt = 0;
        start = end = &pool[cnt++];
    }

    SAMnode &operator [](const int &id){
        return pool[id];
    }

    void push_back(int c, int maxL){
        SAMnode *np = &pool[cnt++], *p = end;
        np->maxL = maxL;
        for (; p && !p->ch[c]; p = p->parent) p->ch[c] = np;
        end = np;
        if (!p){
            np->parent = start;
        } else{
            if (p->ch[c]->maxL == p->maxL + 1){
                np->parent = p->ch[c];
            } else{
                SAMnode *q = &pool[cnt++], *r = p->ch[c];
                *q = *r;
                q->maxL = p->maxL + 1;
                np->parent = r->parent = q;
                for (; p && p->ch[c] == r; p = p->parent) p->ch[c] = q;
            }
        }
    }
} sam;

char str[kMaxN];
int n;
int f[kMaxN * 2], g[kMaxN * 2], v[kMaxN], q[kMaxN * 2];

void init(){
    scanf("%s", str);
    sam.init();
    n = strlen(str);
    for (int i = 0; i < n; i++) sam.push_back(str[i] - 'a', i + 1);
}

void updata(int &x, int y){
    if (x < y) x = y;
}

void getans(char *str){
    memset(g, 0, sizeof(g));
    int n = strlen(str);
    SAMnode *p = sam.start;
    int maxL = 0;
    for (int i = 0; i < n; i++){
        int c = str[i] - 'a';
        if (p->ch[c]){
            p = p->ch[c];
            ++maxL;
            updata(g[p - sam.pool], maxL);
            SAMnode *pp = p->parent;
        } else{
            while (p && !p->ch[c]) p = p->parent;
            if (!p) maxL = 0; else maxL = p->maxL + 1;
            if (!p) p = sam.start; else p = p->ch[c];
            updata(g[p - sam.pool], maxL);
            SAMnode *pp = p->parent;
        }
    }

    for (int i = sam.cnt; i >= 1; i--){
        SAMnode *p = &sam[i];
        if (p->parent) updata(g[p->parent - sam.pool], g[p - sam.pool]);
    }
}

int min(int a, int b){
    return a < b?a:b;
}

void solve(){
    for (int i = 0; i < sam.cnt; i++) v[sam[i].maxL]++;
    for (int i = 1; i <= n; i++) v[i] += v[i - 1];
    for (int i = 0; i <= sam.cnt; i++) q[v[sam[i].maxL]--] = i;
    int ret = 0;
    for (int i = 0; i < sam.cnt; i++) f[i] = sam[i].maxL;
    while (~scanf("%s", str)){
        getans(str);
        for (int i = 0; i < sam.cnt; i++) f[i] = min(f[i], g[i]);
    }
    for (int i = 0; i < sam.cnt; i++) updata(ret, f[i]);
    printf("%d\n", ret);
}

int main(){
    init();
    solve();
    return 0;
}
\end{verbatim}
