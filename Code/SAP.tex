\subsubsection{SAP}
\begin{verbatim}
#include <queue>
#include <cstdio>
#include <cstring>
#include <iostream>
#include <algorithm>
using namespace std;

const int MAXN=102401;
const int MAXM=256000;
const int INF=0x7FFFFFFF;

/*
 * totalNode是网络图总的顶点数
 * S,T分别为源点和汇点
 *
 * 在做SAP之前,先用BFS从T点遍历反向边给每个点距离标号
*/

int n,m,S,T,total,totalNode;
int f[MAXM],next[MAXM],adj[MAXM];
int g[MAXN],last[MAXN],h[MAXN],vh[MAXN];

void ins(int x, int y, int limit){
    f[total]=limit;
    adj[total]=y;
    next[total]=g[x];
    g[x]=total++;

    f[total]=0;
    adj[total]=x;
    next[total]=g[y];
    g[y]=total++;
}

void init(){
    int x,y,limit;
    total=0;
    memset(g,255,sizeof(g));
    totalNode=n;
    scanf("%d%d",&n,&m);
    for (int i=0; i<m; i++){
        scanf("%d%d%d",&x,&y,&limit);
        ins(x,y,limit);
    }
}

int dfs(int node, int add){
    if (node==T) return add;
    int minh=totalNode+1,p=last[node],y,temp;
    do{
        if (f[p]>0){
            y=adj[p];
            if (h[node]==h[y]+1){
                temp=dfs(y,min(add,f[p]));
                if (temp>0){
                    f[p]-=temp;
                    f[p^1]+=temp;
                    last[node]=p;
                    return temp;
                }
            }
            if (h[S]>totalNode) return 0;
            minh=min(minh,h[y]+1);
        }
        p=next[p];
        if (p==-1) p=g[node];
    } while (p!=last[node]);

    if (--vh[h[node]]==0) h[S]=totalNode+1;
    ++vh[h[node]=minh];
    return 0;
}

/*
 * 从汇点T遍历反向边给点标号,如果是无向图就随便遍历
*/

void BFS(int T){
    memset(h,255,sizeof(h));
    memset(vh,0,sizeof(vh));
    vh[h[T]=0]=1;
    queue<int> q;
    q.push(T);
    while (!q.empty()){
        int node=q.front(); q.pop();
        for (int p=g[node]; p!=-1; p=next[p]){
            //p%2==1表示反向边,h[adj[p]]==-1表示该节点未被标号
            if ((p&1) && h[adj[p]]==-1){ 
                ++vh[h[adj[p]]=h[node]+1];
                q.push(adj[p]);
            }
        }
    }
}

int maxFlow(){
	if (g[S]==-1){
		/* no flow */
		return 0;
	}
    BFS(T);
    for (int i=1; i<=totalNode; i++) last[i]=g[i];
    int flow=0;
    while (h[S]<=totalNode) flow+=dfs(S,INF);
    return flow;
}

void solve(){
    cout<<maxFlow()<<endl;
}

int main(){
    init();
    solve();
    return 0;
}
\end{verbatim} 
