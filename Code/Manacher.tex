\subsubsection{回文串Manacher算法}
\begin{verbatim}
Manacher算法的主要思想就是利用对称关系,记录一个对称轴以及回文串长度+对称轴的最大位置,然后算出以当前字符为中心的最长回文串。

#include <string>
#include <cstdio>
#include <cstring>
#include <iostream>
using namespace std;

const int MAXN=100001;

string st;
int ret[MAXN*2];

int Manacher(string &st){
    //偶数位置填一个没有出现的字符
    string str="#";
    for (int i=0; i<st.size(); i++){
        str+=st[i];
        str+='#';
    }
    //ret[i]表示以i这个字符为中心,最长的回文串长度减1
    int MAX=0,pos;
    for (int i=1; i<=str.size(); i++){
        if (MAX>i) ret[i]=min(ret[2*pos-i],MAX-i); else ret[i]=1;
        while (str[i-1-ret[i]]==str[i-1+ret[i]]) ++ret[i];
        if (ret[i]+i>MAX){
            MAX=ret[i]+i;
            pos=i;
        }
    }
    MAX=2;
    for (int i=1; i<=str.size(); i++) MAX=max(MAX,ret[i]);
    return MAX-1;
}

void init(){
    cin>>st;
}

void solve(){
    cout<<Manacher(st)<<endl;
}

int main(){
    init();
    solve();
    return 0;
}
\end{verbatim} 
