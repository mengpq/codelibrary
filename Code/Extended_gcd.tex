\subsection{扩展GCD}
考虑这样一个不定方程:$$ax+by=c$$

易知方程有解的充分必要条件是$(a,b) \mid c$

根据gcd求解的特点,我们可以构造这样一个等式 $$ax+by=bx'+\left (a-b\left \lfloor \frac{a}{b} \right \rfloor\right )y'=c$$

于是有 $$a(x-y')+b\left (y-x'+\left \lfloor \frac{a}{b} \right \rfloor y'\right )=0$$

所以$$x=y'\quad , \quad y=x'-\left \lfloor \frac{a}{b} \right \rfloor y'$$
是方程$ax+by=c$的一个解

扩展gcd递归求解,当$b=0$时,令$x=\frac{c}{a},y=0$是$ax+by=c$的一组特解,然后根据上述等式,我们可以求出方程的一般解。
令$x,y$是方程$ax+by=c$的一组特解,且$d=gcd(a,b)$,则方程$ax+by=c$的一般解为$$X=x+\frac{b}{d}t,Y=y-\frac{a}{d}t \quad |t \in{\mathbf{Z}}$$

我们用$b^{-1}$表示$b$的逆元。因为$\frac{1}{b} \equiv c(mod \quad p) \Rightarrow bx-py=c$
于是我们可以利用扩展gcd可以求出$b$的逆元$b^{-1}$,使得$\frac{a}{b}\equiv c(mod \quad p) \Rightarrow ab^{-1}\equiv c(mod \quad p)$

\begin{verbatim}
程序:
#include<cstdio>
#include<cstring>
#include<iostream>
using namespace std;

int a,b,c,x,y;

void init(){
  cin>>a>>b>>c;
}

void extend_gcd(int a, int b, int &d, int &x, int &y){
    if (!b){
        d=a; x=1; y=0;
    } else{
        extend_gcd(b,a%b,d,y,x);
        y-=x*(a/b);
    }
}

void solve(){
  //求解ax+by=c
  //方程有解的充要条件是gcd(a,b) | c
  //求出方程的一个解x,y,然后对于任意正整数t,
  //方程的一般解可以表示为X=x+b/d*t , Y=y-a/d*t
  extend_gcd(a,b,d,x,y);
  if (c%d!=0) cout<<"No solution"<<endl; else
              cout<<"x="<<x<<" "<<"y="<<y<<endl;
}

int main(){
  init();
  solve();
  return 0;
}
\end{verbatim}
