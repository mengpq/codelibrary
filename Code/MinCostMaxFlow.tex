\subsubsection{费用流}
1、\textbf{求解可行流}: 给定一个网络流图, 初始时每个节点不一定平衡 (每个节点可以有盈余或不足), 每条边的流量可以有上下界, 每条边的当前流量可以不满足上下界约束. 可行流求解中没有源和汇的概念, 算法的目的是寻找一个可以使所有节点都能平衡, 所有边都能满足流量约束的方案, 同时可能附加有最小费用的条件 (最小费用可行流).\par
~\\

2、\textbf{求解最大流}: 给定一个网络流图, 其中有两个特殊的节点称为源和汇. 除源和汇之外, 给定的每个节点一定平衡. 源可以产生无限大的流量, 汇可以吸收无限大的流量. 标准的最大流模型, 初始解一定是可行的 (例如, 所有边流量均为零), 因此边上不能有下界. 算法的目的是寻找一个从源到汇流量最大的方案, 同时不破坏可行约束, 并可能附加有最小费用的条件 (最小费用最大流). \par
~\\
3、\textbf{扩展的最大流}: 在有上下界或有节点盈余的网络流图中求解最大流. 实际上包括两部分, 先是消除下界, 消除盈余, 可能还需要消除不满足最优条件的流量 (最小费用流), 找到一个可行流, 再进一步得到最大流. 因此这里我们的转化似乎是从最大流转化为可行流再变回最大流, 但其实质是将一个过程 (扩展的最大流) 变为了两个过程 (可行流 + 最大流).\par

~\\
4、\textbf{最小费用流的各种转化}
\begin{itemize}
    \item 1.最小费用(可行)流$\to$最小费用最大流
        \begin{itemize}
            \item 建立超级源$s'$和超级汇$t'$,对顶点$i$ ,若$e_i > 0$添加便$s' \to i, c = 0, u = e_i$,若$e_i < 0$添加边$i \to t', c = 0, u = -e_i$,之后求从$s'$到$t'$的最小费用最大流,如果流量等于$\sum{e_i}$,就存在可行流,残量网络已在原图上求出.
        \end{itemize}
    \item 2.最小费用最大流$\to$最小费用(可行)流
        \begin{itemize}
            \item 连边$t\to s, c = -\infty, u = +\infty$,所有点$i$有$e_i = 0$,然后直接球解最小费用最大流.
        \end{itemize}
    \item 3.最小费用(可行)流中负权边的消除
        \begin{itemize}
            \item 直接将负权边满流
        \end{itemize}
    \item 4.最小费用最大流中负权边的消除
        \begin{itemize}
            \item 先连边$t\to s, c = 0, u = +\infty$,使用$(3.)$中的方法消除负权边,使用$(1.)$中的方法求出最小费用(可行)流,之后\textbf{距离标号不变}, 再球最小费用最大流;注意此时增广费用不能机械使用源点的标号,而应该是源点会点标号之差。
        \end{itemize}
\end{itemize}
~\\
\begin{verbatim}
pair<int,int> operator +(const pair<int,int> &a, const pair<int,int> &b){
    return make_pair(a.first + b.first, a.second + b.second);
}

struct netwrok{
    static const int MAX_N = 256;
    static const int MAX_M = 102400;
    static const int INF = 0x3f3f3f3f;

    bool vis[MAX_N];
    int n, cnt, S, T, flow;
    int g[MAX_N], dist[MAX_N], pre[MAX_N];
    int f[MAX_M], adj[MAX_M], next[MAX_M], cost[MAX_M];

    void ins(int x, int y, int limit, int c){
        f[cnt] = limit;
        cost[cnt] = c;
        adj[cnt] = y;
        next[cnt] = g[x];
        g[x] = cnt++;

        f[cnt] = 0;
        cost[cnt] = -c;
        adj[cnt] = x;
        next[cnt] = g[y];
        g[y] = cnt++;
    }

    void init(int _n, int _S, int _T){
        n = _n, S = _S, T = _T;
        cnt = 0;
        memset(g, 255, sizeof(g));
    }

    bool spfa(){
        memset(vis, 0, sizeof(vis));
        memset(dist, 0x3f, sizeof(dist));
        queue<int> q;
        q.push(S);
        vis[S] = true;
        dist[S] = 0;
        while (!q.empty()){
            int node = q.front(); q.pop();
            for (int p = g[node]; p != -1; p = next[p]){
                if (f[p] > 0 && dist[node] + cost[p] < dist[adj[p]]){
                    dist[adj[p]] = dist[node] + cost[p];
                    pre[adj[p]] = p;
                    if (!vis[adj[p]]){
                        q.push(adj[p]);
                        vis[adj[p]] = true;
                    }
                }
            }
            vis[node] = false;
        }
        return dist[T] != INF;
    }

    pair<int,int> augPath(){
        int C = 0, F = INF;
        for (int i = T; i != S; i = adj[pre[i] ^ 1]) F = min(F, f[pre[i]]);
        for (int i = T; i != S; i = adj[pre[i] ^ 1]){
            C += F * cost[pre[i]];
            f[pre[i]] -= F;
            f[pre[i] ^ 1] += F;
        }
        return make_pair(F, C);
    }

    //ret.first -> flow, ret.second -> cosst
    pair<int,int> maxflow(){
        pair<int,int> ret;
        while (spfa()) ret = ret + augPath();
        return ret;
    }
} g;
\end{verbatim}
