\subsection{矩形切割}
\begin{verbatim}
/*
 * 有n个矩形,求这些矩形的面积并
 * 算法的主要思想是枚举第i个矩形,求出第i个矩形被i+1..n的矩形
 * 切割后剩下的面积是多少,加到最终的答案中
 * 
 * 如果题目的每个矩形都有颜色,而且后面的矩形会覆盖前面的矩形,
 * 求每种颜色的面积。这样的题目也可以用矩形切割来做。
*/

//判断两个矩形是否相交,true代表不相交
bool check(int &x1, int &y1, int &x2, int &y2, int &pos){
    return x1>=v[pos].x2 || y1>=v[pos].y2 || x2<=v[pos].x1 || y2<=v[pos].y1;
}

void cut(int x1, int y1, int x2, int y2, int pos){
    while (pos<N && check(x1,y1,x2,y2,pos)) ++pos;

    //已经用v里面的N个矩形切割完
    if (pos==N){
        area+=(x2-x1)*(y2-y1);
        return;
    }

    if (x1<v[pos].x1){
        cut(x1,y1,v[pos].x1,y2,pos+1);
        x1=v[pos].x1;
    }

    if (y1<v[pos].y1){
        cut(x1,y1,x2,v[pos].y1,pos+1);
        y1=v[pos].y1;
    }

    if (x2>v[pos].x2){
        cut(v[pos].x2,y1,x2,y2,pos+1);
        x2=v[pos].x2;
    }

    if (y2>v[pos].y2){
        cut(x1,v[pos].y2,x2,y2,pos+1);
        y2=v[pos].y2;
    }
}

void solve(){
    area=0;
    for (int i=0; i<N; i++)
        cut(v[i].x1,v[i].y1,v[i].x2,v[i].y2,i+1);
    cout<<area<<endl;
}
\end{verbatim}
