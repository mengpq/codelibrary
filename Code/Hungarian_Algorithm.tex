\subsubsection{匈牙利算法}

\begin{verbatim}
/*
 *cx[i]表示X点集中编号为i的点匹配Y点集中编号为cx[i]的点
 *cy[i]表示Y点集中编号为i的点匹配X点集中编号为cy[i]的点
 *sy[i]表示一次找增广路中Y集合编号为i的点是否被访问
*/

#include<cstdio>
#include<vector>
#include<cstring>
#include<iostream>
using namespace std;

const int MAXN=1001;

int n,m;
bool sy[MAXN];
int cx[MAXN],cy[MAXN];
vector<int> g[MAXN];

void init(){
    int x,y;
    cin>>n>>m;
    memset(g,0,sizeof(g));
    for (int i=0; i<m; i++){
        scanf("%d%d",&x,&y);
        g[x].push_back(y);
    }
}

bool find(int k){
    for (int i=0; i<g[k].size(); i++)
    if (!sy[g[k][i]]){
        sy[g[k][i]]=true;
        if (cy[g[k][i]]==0 || find(cy[g[k][i]])){
            cy[g[k][i]]=k;
            cx[k]=g[k][i];
            return true;
        }
    }
    return false;
}

void solve(){
    memset(cx,0,sizeof(cx));
    memset(cy,0,sizeof(cy));
    int ret=0;
    for (int i=1; i<=n; i++){
        memset(sy,0,sizeof(sy));
        if (find(i)) ++ret;
    }
    cout<<ret<<endl;
    for (int i=1; i<=n; i++)
        if (cx[i]!=0)
            cout<<i<<' '<<cx[i]<<endl;
}

int main(){
    init();
    solve();
}
\end{verbatim}
