\subsubsection{Tarjan}
\begin{verbatim}
/*
 * the node is 1-base
*/
#include <vector>
#include <cstdio>
#include <cstring>
using namespace std;

const int MAXN=100001;
const int MAXM=100001;

struct query{
    int node,id;
    query(int node=0, int id=0):node(node),id(id){}
};

int n,m;
bool visit[MAXN];
int lca[MAXM];
int father[MAXN];
vector<int> g[MAXN];
vector<query> q[MAXN];

void init(){
    int x,y;
    memset(g,0,sizeof(g));
    scanf("%d",&n);
    for (int i=1; i<n; i++){
        scanf("%d%d",&x,&y);
        g[x].push_back(y);
        g[y].push_back(x);
    }
}

int find(int k){
    return father[k]!=k?father[k]=find(father[k]):k;
}

void dfs(int node, int fa){
    visit[node]=true;
    for (int i=0; i<g[node].size(); i++)
        if (g[node][i]!=fa){
            dfs(g[node][i],node);
            father[g[node][i]]=node;
        }

    for (int i=0; i<q[node].size(); i++)
        if (visit[q[node][i].node]){
            lca[q[node][i].id]=find(q[node][i].node);
        }
}

void solve(){
    int x,y;
    scanf("%d",&m);
    memset(q,0,sizeof(q));
    for (int i=0; i<m; i++){
        scanf("%d%d",&x,&y);
        q[x].push_back(query(y,i));
        q[y].push_back(query(x,i));
    }
    memset(visit,0,sizeof(visit));
    for (int i=1; i<=n; i++) father[i]=i;
    dfs(1,-1);
    for (int i=0; i<m; i++) printf("%d\n",lca[i]);
}

int main(){
    init();
    solve();
    return 0;
}
\end{verbatim}
