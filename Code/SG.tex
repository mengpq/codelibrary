\subsection{SG函数}
一、对于任意一个Anti-SG游戏,如果我们规定当局面中所有的单一游戏的SG值为0时,游戏结束。
则先手必胜当且仅当:\par

错误形式: \par

(1)所有单一游戏的SG值小于2且游戏的SG值为0 \par

(2)存在单一游戏的SG值大于1且游戏的SG值不为0 \par

正确形式: \par

(1)游戏的SG函数不为0且游戏中某个单一游戏的SG函数大于1 \par

(2)游戏的SG函数为0且游戏中没有单一游戏的SG函数大于1 \par


二、博弈模型 \par

(1)翻硬币游戏:N枚硬币排成一排,有的正面朝上,有的反面朝上。我们开始对硬币按1到N编号。游戏者根据某些约束(如:每次只能翻一枚或两枚,或者每次只能连续的几枚),但他翻动的硬币中,最又边的必须是从正面翻到反面。谁不能翻谁输。
结论:局面的SG值为局面中每个正面朝上的棋子单一存在时的SG值的异或和。\par

(2)树的删边游戏:给出一个有N个点的树,有一个点作为树的根节点。游戏者轮流从树中删去边,删去一条边后,不与跟节点相连的部分将被移走。谁不能操作谁输。
结论:叶子节点的SG值为0;中间节点的SG值为它所有子节点的SG值+1后的异或和。\par

(3)无向图的删边游戏:一个无向连通图,有一个点作为图的根。游戏者轮流从图中删去边,删去一条边后,不与根节点相连的部分被移走。不可操作者输。
