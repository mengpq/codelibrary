\subsubsection{祖先树}
\begin{verbatim}
#include <vector>
#include <cstdio>
#include <cstring>
#include <iostream>
using namespace std;

const int MAXN=100001;
const int maxDeep=17;


/*
 * The ancestor-tree can use to calc LCA with O(logN) Complexity
 * also maintain some information from node i to i's ancestor
 * deep[node] is the distance from root to node
 * father[node][i] is the 2^i ancestor of node
*/

int n,m;
int deep[MAXN],father[MAXN][maxDeep];
vector<int> g[MAXN];

void init(){
    scanf("%d",&n);
    memset(g,0,sizeof(g));
    int x,y;
    for (int i=0; i<n-1; i++){
        scanf("%d%d",&x,&y);
        g[x].push_back(y);
        g[y].push_back(x);
    }
}

void dfs(int node, int fa){
    //update father[node][] use the information calc before
    for (int i=1; i<maxDeep; i++) 
        father[node][i]=father[father[node][i-1]][i-1];

    for (int i=0; i<g[node].size(); i++)
        if (g[node][i]!=fa){
            deep[g[node][i]]=deep[node]+1;
            father[g[node][i]][0]=node;
            dfs(g[node][i],node);
        }
}

int LCA(int x, int y){
    //choose the farther one from root and jump up until x,y has 
    //the same distance from root.
    if (deep[x]<deep[y]) swap(x,y);
    int delta=deep[x]-deep[y];
    for (int i=0; i<maxDeep; i++)
        if (delta & (1<<i)) x=father[x][i];

    //when the x's,y's (2^i)-th father is different,both of them
    //should jump up for 2^i step
    for (int i=maxDeep-1; i>=0; i--)
        if (father[x][i]!=father[y][i]){
            x=father[x][i],y=father[y][i];
        }

    //there are two situtation at last
    return x!=y?father[x][0]:x;
}

void solve(){
    memset(father,0,sizeof(father));
    deep[1]=0;
    dfs(1,0);

    scanf("%d",&m);
    int x,y;
    for (int i=0; i<m; i++){
        scanf("%d%d",&x,&y);
        printf("%d\n",LCA(x,y));
    }
}

int main(){
    init();
    solve();
    return 0;
}
\end{verbatim}
