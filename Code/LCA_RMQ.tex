\subsubsection{LCA-RMQ}
\begin{verbatim}
#include<cmath>
#include<cstdio>
#include<cstring>
#include<iostream>
using namespace std;

struct data {
    int value,pos;
};

const double eps=1e-7;
const int kMaxN=100001;
const int kMaxM=100001;
const int logkMaxN=18;

int n,m,tot,total;
int g[kMaxN],pos[kMaxN];
int adj[kMaxM*2],next[kMaxM*2]; //数组大小*2
data RMQ[kMaxN*2+1][logkMaxN]; //2^(logkMaxN-1)>kMaxN

void connect(int x, int y){
    ++total;
    adj[total]=y;
    next[total]=g[x];
    g[x]=total;
}

void init(){
    scanf("%d%d",&n,&m);
    memset(g,0,sizeof(g));
    total=0;
    int x,y;
    for (int i=1; i<n; i++){
        scanf("%d%d",&x,&y);
        connect(x,y);
        connect(y,x);
    }
}

void dfs(int now, int fa, int deep){
    ++tot;
    pos[now]=tot;
    RMQ[tot][0].value=deep; RMQ[tot][0].pos=now;
    for (int i=g[now]; i!=0; i=next[i])
        if (adj[i]!=fa){
            dfs(adj[i],now,deep+1);
            ++tot;
            RMQ[tot][0].value=deep; RMQ[tot][0].pos=now;
        }
}

void RMQ_CALC(){
    for (int j=1; j<=(int)(log(tot)/log(2)); j++){
        int i;
        for (i=1; i<=tot-(1<<(j-1)); i++)
            if (RMQ[i][j-1].value<RMQ[i+(1<<(j-1))][j-1].value)
                RMQ[i][j]=RMQ[i][j-1]; else
                RMQ[i][j]=RMQ[i+(1<<(j-1))][j-1];
    }
}

void solve(){
    tot=0;
    memset(RMQ,0,sizeof(RMQ));
    dfs(1,0,1);
    RMQ_CALC();
    int x,y;
    for (int i=1; i<=m; i++){
        scanf("%d%d",&x,&y);
        x=pos[x]; y=pos[y];
        if (x>y) swap(x,y);
        //log会产生微小的误差,所以取整前最好要加eps
        int tmp=(int)(log(y-x+1)/log(2)+eps);
        if (RMQ[x][tmp].value<RMQ[y-(1<<tmp)+1][tmp].value)
            printf("%d\n",RMQ[x][tmp].pos); else
            printf("%d\n",RMQ[y-(1<<tmp)+1][tmp].pos);
    }
}

int main(){
    init();
    solve();
}
\end{verbatim} 
