\subsubsection{Dijkstra+heap}
\begin{verbatim}
#include <queue>
#include <cstdio>
#include <cstring>
#include <iostream>
using namespace std;

struct data
{
  int len,node;
  //重载data类型的<
  bool operator < (const data &a) const
  {
    return a.len<len;
  }
};

const int maxn=100001;   //最大顶点数
const int kMaxM=1000001;  //对大边数

int n,m,st,ed,tot;
int g[maxn],dist[maxn],next[kMaxM];
bool vis[maxn];
data adj[kMaxM];

//定义heap q, priority_queue默认是最大堆
priority_queue<data> q;

void insert(int x, int y, int len)
{
  ++tot;
  adj[tot].len=len;
  adj[tot].node=y;
  next[tot]=g[x];
  g[x]=tot;
}

void init()
{
  scanf("%d%d",&n,&m);
  tot=0;
  memset(g,0,sizeof(g));
  int x,y,len;
  for (int i=1; i<=m; i++)
  {
    scanf("%d%d%d",&x,&y,&len);
    insert(x,y,len);
  }
  scanf("%d%d",&st,&ed);
}

void solve()
{
  memset(vis,0,sizeof(vis));
  memset(dist,255,sizeof(dist));
  dist[st]=0;
  //注意,如果是多组数据的话,要用q.pop()清除完q里面的元素
  data tmp; tmp.len=0; tmp.node=st; q.push(tmp);
  while (true)
  {
    tmp=q.top();
    while (!q.empty() && vis[tmp.node])
    {
      q.pop();
      tmp=q.top();
    }
    if (q.empty()) break;
    vis[tmp.node]=true;
    int k=tmp.node, p=g[tmp.node];
    while (p!=0)
    {
      if (dist[adj[p].node]==-1 || dist[k]+adj[p].len<dist[adj[p].node])
      {
        dist[adj[p].node]=dist[k]+adj[p].len;
        tmp.len=dist[adj[p].node]; tmp.node=adj[p].node;
        q.push(tmp);
      }
      p=next[p];
    }
    //如果已经更新到终点ed,则跳出.如果要求起点st到所有点的最短距离的话,忽略之
    //if (vis[ed]) break;
  }
  printf("%d\n",dist[ed]);
  //if (dist[ed]==-1) printf("-1\n"); else printf("%d\n",dist[ed]);
}

int main()
{
  freopen("Dijkstra.in","r",stdin);
  freopen("Dijkstra.out","w",stdout);
    init();
    solve();
  return 0;
}
\end{verbatim} 
