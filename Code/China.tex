\subsection{中国剩余定理}

考虑同余方程组
\begin{displaymath}
\left \{ 
    \begin{array}{l}
        x \equiv a_1(mod \quad m_1) \\
        x \equiv a_2(mod \quad m_2) \\
        \cdots \cdots \\
        x \equiv a_k(mod \quad m_k)
    \end{array} 
\right.
\end{displaymath}

其中$$(m_i,m_j)=1 \quad , \quad 1\leq i,j\leq k,i\neq j$$

设
$$M=m_1m_2\cdots m_k \quad , \quad Mi=\frac{M}{m_i}$$

因为$(Mi,m_i)=1$,所以必然存在$Mi^{-1}$使得$$MiMi^{-1} \equiv 1 (mod \quad m_i)$$ $$MiMi^{-1}\equiv 0 (mod \quad m_j),1\leq j \leq k,j\neq i$$


则$$x=\sum_{i=1}^ka_iMiMi^{-1}$$

是同余方程组模$M$的唯一解,其中$Mi^{-1}$可用扩展gcd求得。

注:同余方程组有解的充要条件是$a_i\equiv a_j(mod \quad (m_i,m_j)),1\leq i,j\leq k$

\begin{verbatim}
程序:
int main(){
  cin>>n;
  for (int i=1; i<=n; i++) cin>>a[i]>>m[i];
  int x , y , M=1 , ans=0;
  //要求m[i]两两互质,同余方程组解唯一
  for (int i=1; i<=n; i++) M*=m[i];
  for (int i=1; i<=n; i++){
    int Mi=M/m[i];
    //求逆元Mi的逆元Mi'
    extended_gcd(Mi,m[i],x,y);
    ans+=a[i]*Mi*x;
  }
  //ans+M*t是一般解
  ans=(ans%M+M)%M;
  cout<<ans<<endl<<endl;
  return 0;
}
\end{verbatim}
