\subsubsection{树的同构}
	【题目】给出一棵$n$个节点的树,用$m$种颜色对其节点染色,问有多少种不同的染色方案。\par
	【思路】首先要处理子树的同构问题,对于任意自同构,质心是不动点(质心:直径的中点,如果直径长度是奇数,则质心在边上)。	\par 
	设$f(T)$表示以树$T$本质不同的染色数,设$r$的儿子中共有$k$个同构等价类,分别为$T_1,T_2,\cdots,T_k$,数量分别为$c_1,c_2,\cdots,c_k$。对于任意等价类分别计算$f(T_i)$,$f(T)$等于不同构的子树染色数的乘积,对于每一个等价类$T_i$,一共有$c_i$个,那么对于这$c_i$个等价类的染色数等价于用$f(T_i)$种颜色对$c_i$个物品染色,这个问题等价于方程$x_1+x_2+\cdots+x_{f(T_i)}=c_i$的方案数$C(f(T_i)+c_i-1,c_i)$。\par
	对于子树的同构判断,我们可以计算每一棵子树$T_i$的hash值。\par
	设$H_i$为树$T_i$的hash值,树$T$有$k$棵子树,分别为$T_1,T_2,\cdots,T_k$,那么树$T$的hash值$H(T)=(((((a*p \mathbf{xor} H_1 mod q)*p \mathbf{xor} H_2) mod q)\cdots*p \mathbf{xor} H_k) mod q)*b mod q$,其中$a,b,p,q$为常量,$H_i$需要从小到大排序。 \par
	~\\
	~\\
\subsubsection{树的分治}
\begin{verbatim}
/*
* active[]表示当前子树的节点
* nodeList存储当前子树的节点
* visit[]表示已经用作根个节点(即已经被计算过)
* data存储当前子树到根的路径的长度以及该路径属于哪一棵子树
* 算法的具体思想就是找当前子树的中心,以中心为根节点,统计子树通过根节点
* 连接起来的长度<=limit的路径。然后再统计删除该根节点后个棵子树的情况,一直递归。
* 时间复杂度O(NlogN),空间复杂度O(n)
*/

#include <vector>
#include <cstdio>
#include <cstring>
#include <iostream>
#include <algorithm>
using namespace std;

struct node_data{
    int node,len;
    node_data(int node=0, int len=0):node(node),len(len){}
};

const int MAXN=100001;

int n,limit;
bool active[MAXN],visit[MAXN];
int total[MAXN];
vector<int> nodeList;
vector<node_data> g[MAXN];
vector< pair<int,int> > data;

void init(){
    int x,y,len;
    memset(g,0,sizeof(g));
    for (int i=1; i<n; i++){
        scanf("%d%d%d",&x,&y,&len);
        --x,--y;
        g[x].push_back(node_data(y,len));
        g[y].push_back(node_data(x,len));
    }
}

int newRoot(int node, int fa, int &root, int &nowMax, int totalNode){
    int maxSize=0,sum=1;
    for (int i=0; i<g[node].size(); i++)
    if (g[node][i].node!=fa && active[g[node][i].node]){
        int temp=newRoot(g[node][i].node,node,root,nowMax,totalNode);
        maxSize=max(maxSize,temp);
        sum+=temp;
    }
    maxSize=max(maxSize,totalNode-sum);
    if (maxSize<nowMax){
        root=node;
        nowMax=maxSize;
    }
    return sum;
}

void getSubtree(int node, int fa, int len, int &part){
    for (int i=0; i<g[node].size(); i++){
        if (active[g[node][i].node] && g[node][i].node!=fa){
            if (g[node][i].len+len>limit) continue;
            data.push_back(make_pair(g[node][i].len+len,part));
            getSubtree(g[node][i].node,node,g[node][i].len+len,part);
        }
    }
}

void getNodeList(int node, int fa){
    active[node]=true;
    nodeList.push_back(node);
    for (int i=0; i<g[node].size(); i++)
    if (g[node][i].node!=fa && !visit[g[node][i].node])
    getNodeList(g[node][i].node,node);
}

long long getAnswer(int node, int fa){
    nodeList.clear();
    getNodeList(node,fa);
    int root,nowMax=n;
    newRoot(node,fa,root,nowMax,nodeList.size());

    int part=0;
    data.clear();
    for (int i=0; i<g[root].size(); i++)
    if (active[g[root][i].node] && g[root][i].len<=limit){
        data.push_back(make_pair(g[root][i].len,part));
        getSubtree(g[root][i].node,root,g[root][i].len,part);
        ++part;
    }
    sort(data.begin(),data.end());

    long long ret=data.size();
    int le=-1,ri=data.size()-1;
    while (ri>0){
        while (le+1<ri && data[le+1].first+data[ri].first<=limit){
            ++le;
            ++total[data[le].second];
        }
        while (le>=ri) --total[data[le--].second];
        ret+=le+1-total[data[ri].second];;
        --ri;
    }
    for (int i=0; i<part; i++) total[i]=0;
    for (int i=0; i<nodeList.size(); i++) active[nodeList[i]]=false;

    visit[root]=true;
    for (int i=0; i<g[root].size(); i++)
    if (!visit[g[root][i].node])
    ret+=getAnswer(g[root][i].node,root);
    return ret;
}

void solve(){
    memset(visit,0,sizeof(visit));
    cout<<getAnswer(0,-1)+n<<endl;
}

int main(){
    int SIZE=1<<23;
    char *p=(char*)malloc(SIZE)+SIZE;
    __asm__("movl %0, %%esp\n" :: "r"(p));
    while (scanf("%d%d",&n,&limit)!=EOF){
        init();
        solve();
    }
    return 0;
}
\end{verbatim}
