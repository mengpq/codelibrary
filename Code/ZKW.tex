\section{ZKW费用流}
\begin{verbatim}
#include <cstdio>
#include <cstring>
#include <deque>
#include <vector>
#include <algorithm>
using namespace std;

#define foreach(it, s) for (__typeof(s.begin()) it = s.begin(); it != s.end(); ++it)

const int MAX_N = 1024;
const int MAX_M = MAX_N * MAX_N;
const int INF = 0x3f3f3f3f;

int cnt = 0, prevCost;

struct    Greap_min_cost_flow {
    int    ecnt;
    struct Edge {
        int to, w, c;
        Edge *next , *other;
    }*mat[MAX_N], edges[MAX_M];
    void link(int x, int to, int w, int c) {
        edges[ecnt].to = to;
        edges[ecnt].w = w;
        edges[ecnt].c = c;
        edges[ecnt].next = mat[x];
        edges[ecnt].other = &edges[ecnt+1];
        mat[x] = &edges[ecnt++];

        edges[ecnt].to = x;
        edges[ecnt].w = 0;
        edges[ecnt].c = -c;
        edges[ecnt].next = mat[to];
        edges[ecnt].other = &edges[ecnt-1];
        mat[to] = &edges[ecnt++];
    }

    int source, remit, piS, cost, f;
    bool v[MAX_N];
    bool check(int &x, int y) {
        if ( x > y ) { x = y; return true; }
        return false;
    }
    bool modlabel_2(){
        int d = INF;
        for(int i=0;i<=remit; ++i)if(v[i])
            for(Edge *p = mat[i]; p; p=p->next)
                if ( p->w && !v[p->to] && p->c<d ) {
                    d = p->c;
                }
        if(d==INF)return false;
        for(int i=0;i<=remit; ++i)if(v[i])
            for(Edge *p = mat[i]; p; p=p->next) {
                p->c -= d;
                p->other->c += d;
            }
        piS += d;
        return true;
    }

    int    aug(int i, int m) {
        if ( i == remit ) {
            cost += piS * m;
            f += m;
            return m; 
        }
        v[i] = 1;
        int l = m;
        for( Edge *p =mat[i]; p ;p = p->next )     
            if ( p->w && !p->c && !v[p->to] ) {
                int d = aug(p->to, min(p->w, l) );
                p->w -= d;
                p->other->w += d;
                l -= d;
                if ( !l ) return m;
            }
        return m-l;
    }
    int clear() {
        ecnt = 0;
        cost = f = 0;
        piS = 0;
        memset(mat, 0, sizeof(mat) );
    }
    int run() {
        memset(v, 0, sizeof(v) );
        cost = 0;
        do {
            do memset(v, 0, sizeof(v) );
            while (aug(source, INF) );
        } while ( modlabel_2() );
        return cost;
    }

    void getvertex(vector< pair<int,int> > &ret){
        ret.clear();
        vector<int> v;
        for (Edge *p = mat[source]; p; p = p->next){
            v.push_back(p->to);
        }
        for (int i = 0; i < v.size(); i++)
            for (Edge *p = mat[v[i]]; p; p = p->next){
                if (p->to != source && p->w == 0) ret.push_back(make_pair(p->to, v[i]));
            }
    }
}G;

int n, m, r, t, k;
int total[MAX_N], starttime[MAX_N];

void init(){
    scanf("%d%d%d%d%d", &n, &m, &r, &t, &k);
    int S = n + m, T = S + 1, x,  y;
    G.clear();
    G.source = S, G.remit = T;
    for (int i = 0; i < m; i++){
        G.link(S, i, 1, 0);
    }

    memset(total, 0, sizeof(total));
    for (int i = 0; i < k; i++){
        scanf("%d%d", &x, &y);
        --x, --y;
        G.link(y, x + m, 1, 0);
        ++total[x];
    }
    for (int i = 0; i < n; i++)
        for (int j = 0; j < total[i]; j++){
            if ((j + 1) * r > t) break;
            G.link(i + m, T, 1, (j + 1) * r);
        }
}

void solve(){
    G.run();
    printf("%d %d\n", G.f, G.cost);
    vector< pair<int, int> > ret;
    G.getvertex(ret);
    memset(starttime, 0, sizeof(starttime));
    foreach(it, ret){
        printf("%d %d %d\n", it->first - m + 1, it->second + 1, starttime[it->first - m]);
        starttime[it->first - m] += r;
    }
}

int main(){
    init();
    solve();
    return 0;
}
\end{verbatim}
