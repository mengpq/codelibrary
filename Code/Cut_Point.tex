\subsubsection{无向图割点}
\begin{verbatim}
/*
 * dfn[]表示访问时的编号,lowlink[]表示不经过父子边能访问到的dfn的最小值
 * 对于v的子节点u,若lowlink[u]>=dfn[v],则v是一个割点。不过对于根节点要特
 * 判,若根节点的子节点个数>1,那么根节点就是割点。
 *
 * 以下算法不能处理重边,如果要处理重边,则需要加一个判断边的id是否被访问
 * 程序默认1为根节点
*/

void dfs(int node, int fa){
    dfn[node]=lowlink[node]=++total;
    for (int i=0; i<g[node].size(); i++)
        if (g[node][i]!=fa){
            if (!dfn[g[node][i]]){
                dfs(g[node][i],node);
                lowlink[node]=min(lowlink[node],lowlink[g[node][i]]);
                if (lowlink[g[node][i]]>=dfn[node]){
                    if (node==1) ++son; else cutPoint[node]=true;
                }
            } else lowlink[node]=min(lowlink[node],dfn[g[node][i]]);
        }
}
\end{verbatim}
