\subsection{Splay}
/* \par
 * Splay多数用来维护一段序列,通常这一段序列又需要reverse操作,或者删除一段子序列等 \par
 * Splay任意节点x提升为节点y的儿子,一般可以在这个操作的基础上完成其他操作 \par
 * 例如要删除一段范围[x,y]的序列,那么可以把x-1这个节点提升为根,再把y+1这个节点提升 \par
 * 为x-1的儿子(右儿子),然后y+1这个节点的左儿子就是[x,y]这一段序列 \par
 * \par
 * Splay的操作通常都是把一段范围的序列变为某个节点的一棵子树,然后就可以对这棵子树 \par
 * 进行操作 \par
 * \par
 * 因为空序列不好处理,所以Splay一般都会加入两个不影响结果的辅助节点(头、尾),至于 \par
 * 怎样不影响结果,要根据题目来设定。 \par
 * \par
 * Splay提升某个节点的时候,有两种情况。一种是先调用select得到节点x,然后再提升x,这样 \par
 * 在splay操作时,就不需要pushdown标签,因为在select过程中已经pushdown了。 \par
 * 但如果是已知某个节点x,需要splay提升,这是在splay提升的过程中,就要注意pushdown标签了 \par
 * \textbf{在对Splay操作时,要时刻注意是否要进行pushdown操作} \par
 * \par
 * \par
 * root               :\textbf{SplayTree的根节点,要注意时刻维护!!!} \par
 * build(data,l,r,fa) :如果初始时给出序列,那么可以用build来建立一棵平衡的SplayTree \par
 * rotate(x)          :将节点x往上旋转(左右旋转都没问题) \par
 * splay(x,fa)           :将节点x提升为fa的儿子(当fa=NULL即将x提升为根) \par
 * select(x,k)           :一般调用为select(root,k),即获得第排名为k的节点,在这个过程 \par
 *                        中需要先pushdown下传标签 \par
 * join(x,y)          :合并x,y两棵SplayTree,其中y在x的右边。首先将x最右的节点提升为 \par
 *                        x的根,此时x无右子树,所以直接将y连接为x的右子树 \par
 * del(k)               :删除排名为k的节点。首先将第k个节点提升为根,然后直接合并根的两棵 \par
 *                         子树,注意任意时刻Splay都最少保持有2个节点(开始加入的2个节点) \par
 * insert(key,k)      :在第k个节点前插入一个key,首先将节点k-1提升为根,然后将节点k提升 \par
 *                      为k-1的儿子,此时节点k无左子树,然后将key插入到k的左子树中。 \par
 *                         要在第k个节点之前插入一段连续的数的操作和一个数差不多。 \par
*/ \par
~\\

题目:reginoal 2012 HangZhou A problem \par
1: add x \par
Starting from the arrow pointed element, add x to the number on the clockwise first k2 elements.\par
 \par
2: reverse \par
Starting from the arrow pointed element, reverse the first k1 clockwise elements. \par
 \par
3: insert x  \par
Insert a new element with number x to the right (along clockwise) of the arrow pointed element. \par
 \par
4: delete  \par
Delete the element the arrow pointed and then move the arrow to the right element. \par
 \par
5: move x  \par
x can only be 1 or 2. If x = 1 , move the arrow to the left(along the counterclockwise) element, if x = 2 move the arrow to the right element. \par
 \par
6: query \par
Output the number on the arrow pointed element in one line. \par

\begin{verbatim}
#include <stdio.h>
#include <string.h>
#include <algorithm>
using namespace std;

#define REP(i,st,ed) for (int i=st; i<ed; i++)

const int kMaxN = 102400;

struct splayNode{
    splayNode *fa,*ch[2];
    bool rev;
    int size,key,add;
    splayNode(int key=0){
        this->key=key;
        size=1,add=0,rev=false;
        fa=ch[0]=ch[1]=NULL;
    }
    void clear(){
        if (ch[0]) ch[0]->clear();
        if (ch[1]) ch[1]->clear();
        delete this;
    }
    void update(){
        size=(ch[0]?ch[0]->size:0)+(ch[1]?ch[1]->size:0)+1;
    }
    void down(){
        if (add){
            if (ch[0]) ch[0]->key+=add,ch[0]->add+=add;
            if (ch[1]) ch[1]->key+=add,ch[1]->add+=add;
            add=0;
        }
        if (rev){
            swap(ch[0],ch[1]);
            if (ch[0]) ch[0]->rev^=1;
            if (ch[1]) ch[1]->rev^=1;
            rev=false;
        }
    }
    splayNode* rightmost(){
        down();
        if (ch[1]) return ch[1]->rightmost();
        return this;
    }
} *root,*temp;

typedef splayNode* ptr;

int n,m,k1,k2,data[kMaxN];

ptr build(int data[], int l, int r, ptr fa=NULL){
    int mid=(l+r)/2;
    ptr node=new splayNode(data[mid]);
    node->fa=fa;
    if (l<mid) node->ch[0]=build(data,l,mid-1,node);
    if (mid<r) node->ch[1]=build(data,mid+1,r,node);
    node->update();
    return node;
}

void rotate(ptr x){
    ptr y=x->fa;
    if (x->fa=y->fa) y->fa->ch[y->fa->ch[1]==y]=x;
    int o=y->ch[0]==x;
    if (y->ch[!o]=x->ch[o]) x->ch[o]->fa=y;
    x->ch[o]=y;
    y->fa=x;
    y->update();
    x->update();
}

void splay(ptr x, ptr fa){
    while (x->fa!=fa){
        /* this condition are need sometime
         * if (x->fa->fa) x->fa->fa->down();
         * if (x->fa) x->fa->down();
         * x->down();
        */
        if (x->fa->fa==fa) rotate(x); else{
            int o1=x->fa->ch[0]==x,o2=x->fa->fa->ch[0]==x->fa;
            if (o1==o2) rotate(x->fa),rotate(x); else rotate(x),rotate(x);
        }
    }
}

ptr select(ptr x, int k){
    x->down();
    if (x->ch[0]){
        if (x->ch[0]->size>=k) return select(x->ch[0],k);
        k-=x->ch[0]->size;
    }
    if (k==1) return x;
    return select(x->ch[1],k-1);
}

ptr join(ptr x, ptr y){
    ptr temp=x->rightmost();
    splay(temp,NULL);
    temp->ch[1]=y;
    y->fa=temp;
    temp->update();
    return temp;
}

void del(int k){
    root=select(root,k);
    splay(root,NULL);
    root->ch[0]->fa=root->ch[1]->fa=NULL;
    root=join(root->ch[0],root->ch[1]);
}

void insert(int key, int k){
    root=select(root,k-1);
    splay(root,NULL);
    temp=select(root,k);
    splay(temp,root);
    temp->ch[0]=new splayNode(key);
    temp->ch[0]->fa=temp;
    temp->update();
    root->update();
}

void init(){
    REP(i,1,n+1) scanf("%d",&data[i]);
    root=build(data,0,n+1);
}

void solve(){
    char st[10];
    int number;
    REP(i,0,m){
        scanf("%s",st);
        if (st[0]=='a'){
            scanf("%d",&number);
            root=select(root,1);
            splay(root,NULL);
            temp=select(root,k2+2);
            splay(temp,root);
            temp->ch[0]->key+=number;
            temp->ch[0]->add+=number;
        } else if (st[0]=='r'){
            root=select(root,1);
            splay(root,NULL);
            temp=select(root,k1+2);
            splay(temp,root);
            root->ch[1]->ch[0]->rev^=1;
        } else if (st[0]=='i'){
            scanf("%d",&number);
            insert(number,3);
        } else if (st[0]=='d'){
            del(2);
        } else if (st[0]=='m'){
            scanf("%d",&number);
            if (number==1){
                temp=select(root,root->size-1);
                int key=temp->key;
                del(root->size-1);
                insert(key,2);
            } else{
                temp=select(root,2);
                int key=temp->key;
                del(2);
                insert(key,root->size);
            }
        } else if (st[0]=='q'){
            root=select(root,2);
            splay(root,NULL);
            printf("%d\n",root->key);
        }
    }
    root->clear();
}

int main(){
    int ca=0;
    while (scanf("%d%d%d%d",&n,&m,&k1,&k2)){
        if (n==0 && m==0 && k1==0 && k2==0) return 0;
        printf("Case #%d:\n",++ca);
        init();
        solve();
    }
    return 0;
}
\end{verbatim}
