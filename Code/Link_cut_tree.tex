\subsection{树链剖分}
\begin{verbatim}
/*
 * hson[i]表示i这个节点所连接的那条重边对应的子节点编号
 *
 * top[i]表示i这个节点所属的那条链的链头节点编号
 * 
 * idx[i]表示编号唯i的节点在线段树中的编号,一条链在线段树上的编号是连续的
 *
 * 树链剖分的步骤:
 * 1、遍历整棵树,找子树节点个数最大的子节点作为主链
 * 2、遍历轻重边,把原树上的节点与其在线段树的节点对应
 * 记录每条链在线段树中的位置以及链头所链接的父亲节点
 * 3、在询问路径时,每次两个节点深度较大的网上跳,直到两个节点在同一条链上为止。
 *
 * 每次询问的时间复杂度为O((logN)^2)
 *
 * 本程序的例题:询问路径中节点的最大值。把某个节点的值
 * 增加一个值。
 *
*/
#pragma comment(linker, "/STACK:16777216")

#include <cstdio>
#include <cstring>
#include <vector>
#include <algorithm>
using namespace std;

const int kMaxN = 100001;

int n,totalSize;
int size[kMaxN],top[kMaxN],depth[kMaxN],idx[kMaxN],hson[kMaxN],parent[kMaxN];
int tree[kMaxN * 4];
vector<int> g[kMaxN];

void init(){
    int x,y;
    scanf("%d",&n);
    for (int i = 1; i <= n; i++) g[i].clear();
    for (int i = 1; i < n; i++){
        scanf("%d%d",&x,&y);
        g[x].push_back(y);
        g[y].push_back(x);
    }
}

void dfs(int node){
    size[node] = 1, hson[node] = -1;
    for (int i = 0; i < g[node].size(); i++) if (g[node][i] != parent[node]){
        int son = g[node][i];
        depth[son] = depth[node] + 1;
        parent[son] = node;
        dfs(son);
        size[node] += size[son];
        if (hson[node] == -1 || size[son] > size[hson[node]]) hson[node] = son;
    }
}

void build(int node, int tp){
    idx[node] = ++totalSize; 
    top[node] = tp;
    if (hson[node] != -1) build(hson[node], tp);
    for (int i = 0; i < g[node].size(); i++){
        int son = g[node][i];
        if (son != hson[node] && son != parent[node]) build(son,son);
    }
}

void change(int x, int y, int state, int L, int R, int value){
    if (L > y || x > R) return;
    if (L <= x && y <= R){
        tree[state] += value;
        return;
    }

    int mid = (x + y) / 2;
    change(x, mid, state * 2, L, R, value);
    change(mid + 1, y, state * 2 + 1, L, R, value);
    tree[state] = max(tree[state * 2], tree[state * 2 + 1]);
}

int ask(int x, int y, int state, int L, int R){
    if (L > y || x > R) return 0;
    if (L <= x && y <= R) return tree[state];

    int mid = (x + y) / 2;
    return max(ask(x, mid, state * 2, L, R), ask(mid + 1, y, state * 2 + 1, L, R));
}

int query_max(int x, int y){
    int tpx = top[x], tpy = top[y], ret = 0;
    while (tpx != tpy){
        if (depth[tpx] < depth[tpy]) swap(tpx,tpy), swap(x,y);
        ret = max(ret, ask(1, n, 1, idx[tpx], idx[x]));
        x = parent[tpx], tpx = top[x];
    }
    if (depth[x] > depth[y]) swap(x,y);
    ret = max(ret, ask(1, n, 1, idx[x], idx[y]));
    return ret;
}

void solve(){
    dfs(1);
    totalSize = 0;
    memset(tree,0,sizeof(tree));
    build(1,1);

    char st[3];
    int m,x,y,value;
    scanf("%d",&m);
    for (int i = 0; i < m; i++){
        scanf("%s",st);
        if (st[0] == 'I'){
            scanf("%d%d",&x,&value);
            change(1, n, 1, idx[x], idx[x], value);
        } else{
            scanf("%d%d",&x,&y);
            printf("%d\n",query_max(x,y));
        }
    }
}

int main(){
    init();
    solve();
    return 0;
}
\end{verbatim}
