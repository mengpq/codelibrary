\subsection{树链剖分}
\begin{verbatim}
/*
 * nextNode[i]表示树链剖分时第i个节点的下一个节点,如果没有则为-1
 *
 * chain[i-th]存储第i条链的三个信息:链头的父亲节点、链头及链尾节
 * 点在线段树中的编号。
 *
 * nodeInfo[i]存储树中第i个节点的两个信息:在线段树中的编号pos,
 * 所属的链的编号part。
 *
 * 树链剖分的步骤:
 * 1、遍历整棵树,找子树节点个数最大的子节点作为主链。
 * 2、遍历轻重边,把原树上的节点与其在线段树的节点对应。
 * 记录每条链在线段树中的位置以及链头所链接的父亲节点。
 * 3、在询问路径时,从一个节点一直往LCA走,直到查询节点
 * 和LCA在同一条链为止。
 *
 * 每次询问的时间复杂度为O((logN)^2)
 *
 * 本程序的例题:询问路径中节点的最大值。把某个节点的值
 * 增加一个值。
 *
*/

#include <vector>
#include <cstdio>
#include <cstring>
#include <algorithm>
using namespace std;

#pragma comment(linker, "/STACK:16777216")

const int MAXN=110001;
const int MAXM=110001;

struct query{
    int node,id;
    query(int node=0, int id=0):node(node),id(id){}
};

struct opt_data{
    char ch;
    int U,V;
};

struct node_data{
    int pos,part;
};

struct chain_data{
    int father,le,ri;
};

int n,m;
bool visit[MAXN];
int lca[MAXM];
int father[MAXN],tree[4*MAXN],nextNode[MAXN];
opt_data opt[MAXM];
vector<int> graph[MAXN];
vector<query> q[MAXN];
chain_data chain[MAXN];
node_data nodeInfo[MAXN];

void init(){
    int x,y;
    memset(graph,0,sizeof(graph));
    scanf("%d\n",&n);
    for (int i=1; i<n;i++){
        scanf("%d%d\n",&x,&y);
        graph[x].push_back(y);
        graph[y].push_back(x);
    }
    
    memset(q,0,sizeof(q));
    scanf("%d\n",&m);
    for (int i=0; i<m; i++){
        scanf("%c %d %d\n",&opt[i].ch,&opt[i].U,&opt[i].V);
        if (opt[i].ch=='G'){
            q[opt[i].U].push_back(query(opt[i].V,i));
            q[opt[i].V].push_back(query(opt[i].U,i));
        }
    }
}

int find(int k){
    int x,y=k;
    while (father[y]!=y) y=father[y];
    while (k!=y){
        x=father[k];
        father[k]=y;
        k=x;
    }
    return y;
}

/*
 * Travel the tree and mark chain's node select one node 
 * with max subtree size,add this node to the main chain.
 * Calculate the LCA for each query at the meantime.
*/
int dfs(int node){
    visit[node]=true;
    int maxSize=0,selectNode,sum=0;
    for (int i=0; i<graph[node].size(); i++)
        if (!visit[graph[node][i]]){
            int temp=dfs(graph[node][i]);
            father[graph[node][i]]=node;
            sum+=temp;
            if (temp>maxSize){
                maxSize=temp;
                selectNode=graph[node][i];
            }
        }
    if (maxSize>0){
        nextNode[node]=selectNode;
    }
    for (int i=0; i<q[node].size(); i++)
        if (visit[q[node][i].node])
            lca[q[node][i].id]=find(q[node][i].node);
    return sum+1;
}

/*
 * Travel the tree again,mapping the tree's node onto
 * the segment-tree.
*/
void travel(int node, int &totalPart, int &totalNode){
    visit[node]=true;
    nodeInfo[node].pos=++totalNode;
    nodeInfo[node].part=totalPart;
    if (nextNode[node]==-1){
        chain[totalPart].ri=totalNode;
    } else{
        travel(nextNode[node],totalPart,totalNode);
    }

    for (int i=0; i<graph[node].size(); i++)
        if (!visit[graph[node][i]]){
            ++totalPart;
            chain[totalPart].father=node;
            chain[totalPart].le=totalNode+1;
            travel(graph[node][i],totalPart,totalNode);
        }
}

int ask(int l, int r, int state, int &L, int &R){
    if (L>r || l>R) return -0x7FFFFFFF;
    if (L<=l && r<=R) return tree[state];

    int mid=(l+r)/2;
    return max(ask(l,mid,state*2,L,R),ask(mid+1,r,state*2+1,L,R));
}

void change(int l, int r, int state, int &L, int &R, int &value){
    if (L>r || l>R) return;
    if (L<=l && r<=R){
        tree[state]+=value;
        return;
    }
    
    int mid=(l+r)/2;
    change(l,mid,state*2,L,R,value);
    change(mid+1,r,state*2+1,L,R,value);
    tree[state]=max(tree[state*2],tree[state*2+1]);
}

int ask_max(int node, int LCA){
    int ret=-0x7FFFFFFF;
    while (nodeInfo[node].part!=nodeInfo[LCA].part){
        int part=nodeInfo[node].part,le=chain[part].le,ri=nodeInfo[node].pos;
        if (le>ri) swap(le,ri);
        ret=max(ret,ask(1,n,1,le,ri));
        node=chain[part].father;
    }
    int le=nodeInfo[node].pos,ri=nodeInfo[LCA].pos;
    if (le>ri) swap(le,ri);
    ret=max(ret,ask(1,n,1,le,ri));
    return ret;
}

void solve(){
    memset(visit,0,sizeof(visit));
    memset(nextNode,255,sizeof(nextNode));
    for (int i=1; i<=n; i++) father[i]=i;
    dfs(1);

    memset(visit,0,sizeof(visit));
    int totalPart=1,totalNode=0;
    chain[1].le=1;
    travel(1,totalPart,totalNode);

    memset(tree,0,sizeof(tree));
    for (int i=0; i<m; i++){
        if (opt[i].ch=='G'){
            printf("%d\n",max(ask_max(opt[i].U,lca[i]),
                    ask_max(opt[i].V,lca[i])));
        } else{
            change(1,n,1,nodeInfo[opt[i].U].pos,
                    nodeInfo[opt[i].U].pos,opt[i].V);
        }
    }
}

int main(){
    init();
    solve();
    return 0;
}
\end{verbatim}
