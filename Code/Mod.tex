\subsection{组合数取模}
如果模数为素数,直接求逆元就可以了。否则就把模数因式分解成$\prod_{i=1}^mp_i^{k_i}$,求出组合数模$p^{k_i}$的答案,然后用中国剩余定理把这些解合并。
如果要求$t$个组合数数$C(n,m)$的对模$Mod$的模,那么总的时间复杂度是$O(t(logn)^2)$级别的。

\begin{verbatim}
#include <cstring>
#include <cstdio>
#include <vector>
#include <iostream>
#include <algorithm>
using namespace std;

const int MAXN=100001;

struct data{
    int Pi,P;
    data(){}
    data(int Pi, int P):Pi(Pi),P(P){}
};

struct com{
    int n,m;
    com(){}
    com(int n, int m):n(n),m(m){}
};

int n,m,ca,Mod;
vector<com> ask;
long long factorial[MAXN],power[2];

void init(){
    int x,y;
    scanf("%d%d%d",&n,&m,&Mod);
    ask.clear();
    for (int i=0; i<m; i++){
        scanf("%d%d",&x,&y);
        ask.push_back(com(x,y));
    }
}

void extended_gcd(int a, int b, int &x, int &y){
    if (!b){
        x=1,y=0;
    } else{
        extended_gcd(b,a%b,y,x);
        y-=(a/b)*x;
    }
}

//calc 1/a mod b
int calcInv(int a, int b){
    int x,y;
    extended_gcd(a,b,x,y);
    return (x%b+b)%b;
}

vector<data> factorization(int Mod){
    vector<data> factor;
    int N=Mod;
    for (int i=2; i*i<=Mod; i++)
        if (N%i==0){
            int temp=1;
            while (N%i==0){
                temp*=i;
                N/=i;
            }
            factor.push_back(data(i,temp));
        }
    if (N!=1) factor.push_back(data(N,N));
    return factor;
}

void initialization(int Pi, int P){
    factorial[0]=1;
    for (int i=1; i<min(MAXN,P); i++){
        if (i%Pi!=0){
            factorial[i]=factorial[i-1]*i%P;
        } else factorial[i]=factorial[i-1];
    }
    if (P<MAXN){
        power[1]=factorial[P-1]; power[0]=power[1]*power[1]%P;
    }
}

//calc n! mod P
int calcFactorialMod(int n, int &cnt, int Pi, int P){
    cnt=0; long long ret=1;
    for (int i=n; i>=1; i/=Pi){
        ret=ret*power[(i/P)&1]*factorial[i%P]%P;
        cnt+=i/Pi;
    }
    return ret;
}

//calculate C(n,m)%P
int combination(int n, int m, int Pi, int P){
    if (m>n) return 0;
    int x,y,cnt1,cnt2,cnt3;
    long long ret1=calcFactorialMod(n,cnt1,Pi,P);
    long long ret2=calcFactorialMod(m,cnt2,Pi,P);
    long long ret3=calcFactorialMod(n-m,cnt3,Pi,P);

    cnt1-=cnt2+cnt3;
    ret1=ret1*calcInv(ret2*ret3%P,P)%P;
    for (int i=0; i<cnt1; i++) ret1=ret1*Pi%P;
    return ret1;
}

//merge
pair<int,int> Chinese_remainder(pair<int,int> a, pair<int,int> b){
    long long p1=a.second,p2=b.second,M=p1*p2;
    int t1=p2*calcInv(p2,p1)*a.first%M,t2=p1*calcInv(p1,p2)*b.first%M;
    return make_pair((t1+t2)%M,M);
}

void solve(){
    vector<data> factor=factorization(Mod);
    vector< pair<int,int> > final(ask.size());
    for (int i=0; i<final.size(); i++) final[i].first=final[i].second=-1;
    for (int i=0; i<factor.size(); i++){
        initialization(factor[i].Pi,factor[i].P);
        for (int j=0; j<ask.size(); j++){
            pair<int,int> p=make_pair(combination(ask[j].n,ask[j].m,  
                        factor[i].Pi,factor[i].P),factor[i].P);
            if (final[j].first==-1){
                final[j]=p;
            } else{
                final[j]=Chinese_remainder(p,final[j]);
            }
        }
    }
    for (int i=0; i<final.size(); i++)
        printf("C(%d,%d) Mod %d = %d\n",ask[i].n,ask[i].m,Mod,final[i].first);
}

int main(){
    scanf("%d",&ca);
    for (int i=0; i<ca; i++){
        init();
        solve();
    }
    return 0;
}
\end{verbatim}

