	\subsection{GSS1}
	\subsubsection{题目大意} \par
	给出一个序列$A_1,A_2,\ldots,A_N.(|A_i|\leq 15007,N\leq 50000)$。\par
	然后给出$M$个询问,每个询问给出$x,y$两个下标,然后求区间$[x,y]$的最大连续子段和是多少。\par
	\subsubsection{解题思路} \par
	\textbf{方法一:}线段树的区间记录$3$个信息$\mathbf{max,lMax,rMax}$分别表示当前区间的最大子段和、以区间最左边为起始点的最大子段和以及以区间最右边为终点的最大子段和。查询的时候,答案是左子树最大值、右子树最大值、以及区间中间部分。\par
	方法是很显然的,但如果求区间中间部分的最大连续和实现不好的话,就增加了编程复杂度。一个比较好的实现方法就是在询问的时候,先询问完左右子树(如果只有一个的话就访问一个),然后合并左右子树得到在查询区间内的一棵线段树,这棵线段树同样业维护那$3$个信息。这样就避免了讨论中间区间的最值是否会越界的问题了。\par
	这个方法的时间复杂度是$O(MlogN)$
	~\\
	~\par
	\textbf{方法二:}可以用分块的思想。按照每个块$\sqrt N$的长度把$N$个数分成$\sqrt N$块,然后每个块维护的信息和方法一的一样。\par
	这个方法的时间复杂度是$O(M\sqrt N)$

	\subsection{GSS2}
	\subsubsection{题目大意} \par
	给出$N(N\leq 100000)$个数,然后有$Q(Q\leq 100000)$个询问,每个询问给出两个整数$x,y(x,y\leq N)$,问区间$[x,y]$中最大的连续和,其中相同的数只算一次。
	\subsubsection{解题思路} \par
	设$pre[i]=max(j)\ \Big |(a_i=a_j\ \& \&\ 1\leq j<i) $,若不存在这样的$j$则$pre[i]=0$。\par
	先考虑这样一个问题:求区间$[x,y]$中不同的数的和。\par
	事实上,这个问题等价于求区$[x,y]$中满足$pre[i]<x$的$a_i$的和。如果我们将一段区间内的$pre[]$有序,那么这个问题就可以得到解决。但是,如果让$pre[]$有序,就打乱了序列原来的顺序,不利于求连续一段区间的最大连续和。所以要换一个求解模型。\par
	设$f[i]$表示为从第$i$个数开始的最大连续和。我们从第一个数开始,一个个数按顺序插入序列中,对于第$k$个数,它的加入能影响的范围为$i \in [pre[k]+1,k]$的$i$。维护$f[i]$类似维护不重复最大连续和一样。然后用线段树维护$f[i]$,要用到线段树的传递性,具体描述起来很麻烦,但是比较容易想。


	\subsection{GSS3}
	\subsubsection{题目大意}
		和GSS1一样,只是多了一个修改操作,而且是只修改一个数。\par
	\subsubsection{解题思路}
		和GSS1一样,修改操作的维护也很容易。



	\subsection{GSS4}
	\subsubsection{题目大意}
	给出$N(N\leq 100000)$个正数,这些数的总和不超过$10^{18}$,然后有$M(M\leq 100000)$个操作,每个操作为以下两种之一:\par
	1、给出两个数$x,y(x,y\leq N)$,对于$i\in [x,y]$,将$a_i$变为$\lfloor \sqrt a_i \rfloor$。 \par
	2、给出两个数$x,y(x,y\leq N)$,求区间$[x,y]$的数的总和。\par

	\subsubsection{解题思路}
	首先可以观察到对于$i\in [1,n]$,有$a_i<10^{18}$,而且对同一个$a_i$开平方根,10次以内就可以收敛于1。这就以为着,即使对于每一次我们都暴力地把区间的数开根,修改的次数大概是$10^6$级别的。\par
	如果用线段树解决这个求和问题的话,那么在线段树中维护一个$allOne$的布尔值,表示当前区间的数是否全为1。当一个区间的数全为1时,修改操作对于这个区间是无意义的,所以不必往下修改。\par
	第二种做法,令$next[i]=min(j)\Big |(a_i\neq 1\ \& \& \ i\leq j\leq N)$,我们可以用并查集维护$next[i]$,这样就可以做到每次修改都只修改那些不为1的数。然后求区间的和可以用树状数组动态维护前缀和,两个前缀和相减就是答案。\par



\subsection{GSS5}
	\subsubsection{题目大意} \par
	给出$N(N\leq 10000)$个数。 \par
	有$M(M\leq 10000)$个询问,每个询问给出两个区间$[x_1,y_1],[x_2,y_2]$,求一段最大连续和$[x,y]$,使得$x\in [x_1,y_1],\ y\in [x_2,y_2]$。 \par
	\subsubsection{解题思路} \par
	设$f[i]$表示前$i$个数的和。\par
	1、若$[x_1,y_1]$和$[x_2,y_2]$不相交,那么最大值就是$\mathbf{max}(f[y])-\mathbf{min}(f[x])\Big | x \in [x_1-1,y_1-1],\ y\in [x_2,y_2]$。\par
	2、分三种情况,第一种考虑$x\in[x_1,x_2],\ y\in[y_1,y_2]$,此时的最大值是$\mathbf{max}(f[y])-\mathbf{min}(f[x])\Big | x\in [x_1-1,x_2-1],y\in [y_1,y_2]$。第二种考虑$x\in[x_1,y_1],\ y\in[y_1,y+2]$,此时的最大值是$\mathbf{max}(f[y])-\mathbf{min}(f[x])\Big | x\in [x_1-1,y_1-1],\ y\in [y_1,y_2]$。第三种考虑$x\in[x_2,y_1],y\in[x_2,y_1]$,此时问题和GSS1一样。\par
	这3种情况都可以用线段树维护。总的时间复杂度应该是$O(MlogN)$的,不知到为什么数据范围给那么小。


	\subsection{GSS6}
	\subsubsection{题目大意} \par
	给出一个序列,然后对这个序列进行询问最大连续和、修改、删除等操作。
	\subsubsection{解题思路} \par
	和GSS1相比就多了修改和删除操作。这里有删除操作,于是就用平衡树去维护。类似线段树那样记录几个数据就可以了。另外,用Splay来写就方便很多。


	\subsection{GSS7}
	\subsubsection{题目大意} \par
	给出一棵有$N(N\leq100000)$个节点的树,树上的节点都有一个值$x_i(|x_i|\leq 10000$。\par
	然后有$Q(Q\leq 100000)$个操作,每个操作是以下两种之一:\par
	1、\textsl{1 a b},表示询问节点$a$到节点$b$路径的最大连续和。\par
	2、\textsl{2 a b c},表示把节点$a$到节点$b$路径中所有点的值改成$c$。\par
	\subsubsection{解题思路} \par
	先做树链剖分,对于每一条链用线段树维护$\mathbf{sum,max,lMax,rMax}$。\par
	对于每次询问,从两个节点分别往$\mathbf{LCA}$走,求出最大值$\mathbf{ret}$。类似$\mathbf{lMax,rMax}$那样,同时也要分别记录两个节点到$\mathbf{LCA}$的最大值$\mathbf{xUpMax,yUpMax}$。如果$\mathbf{xUpMax,yUpMax}$都不为0,那么最大值为$max(\mathbf{ret,xUpMax+yUpMax-value)}$,因为$\mathbf{LCA}$的值被计算了两次,所以要减去一次$\mathbf{LCA}$的值$value$。\par
	对于修改操作,由于操作涉及修改一段数变为一个数,这个只要用线段树的传递标签$lazy-tag$即可。\par
	每次操作的时间复杂度为$O((logN)^2)$,所以总的时间复杂度为$O(NlogN+M(logN)^2)$


	
