\def\ChineseScale{1200}
\documentclass[12pt,a4paper,titlepage]{article}
\usepackage{times}
\usepackage{CJK}
%CJK中文支持
\usepackage{latexsym,bm}
\usepackage{indentfirst}
%自动首行缩进
\usepackage[CJKbookmarks,colorlinks=true]{hyperref}
%CJKbookmarks支持中文标签, colorlinks=true 就是把超链接的边框去掉,但字是红色的
\usepackage[top=1in, bottom=1in, left=0.5in, right=0.5in]{geometry}
%用来调页边距

%\author{AcFast}
%\title{title}

\begin{document}
\begin{CJK}{UTF8}{gbsn}
	若二次方程$x^2\equiv a(\mathbf{mod}\ p)$存在解,则称$a$为$p$的二次剩余,否则称$a$为$p$的二次非剩余。
	当$p$为奇素数时,有$\mathbf{Legendre}$符号: \par
	\[ \Big (\frac{a}{p} \Big )= \left\{ 
		\begin{array}{ll}
			1  & \mbox{$a$为$p$的二次剩余} \\
			-1 & \mbox{$a$为$p$的二次非剩余} \\
			0  & \mbox{$a\equiv 0(\mathbf{mod}\  p)$} \\
		\end{array} \right. \] 
	$\mathbf{Legendre}$符号的一些性质:\par
	1、$\Big (\frac{a}{p} \Big )=a^{(p-1)/2}$ \par
	2、$\Big (\frac{a+p}{p} \Big )=\Big (\frac{a}{p}\Big )$ \par
	3、$\Big (\frac{ab}{p} \Big ) =\Big (\frac{a}{p} \Big )\Big ( \frac{b}{p} \Big )$ \par
	3、$\Big (\frac{-1}{p} \Big ) =(-1)^{\frac{p^2-1}{8}}$
\end{CJK}
\end{document}


