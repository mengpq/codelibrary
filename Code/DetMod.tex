\subsection{行列式取模}

\subsubsection{辗转相除法}
\begin{verbatim}
/*
 * 1、若矩阵两行进行交换,行列式取反
 * 2、一行数加减另外一行数的倍数,行列式不变
 * 于是对于第i行和第j行,可以将第j行减去matrix[j][i]/matrix[i][i]倍的i行
 * 然后把两行交换,不断迭代至第j行的第i个元素变成0为止
 * 这个方法适用于任何模数,时间复杂度为O(n^3logC),C就是矩阵的数的范围
*/
int gauss(int matrix[][kMaxN], int n){
    int ret=1;
    for (int i=0; i<n; i++)
        for (int j=0; j<n; j++)
            matrix[i][j]=(matrix[i][j]%Mod+Mod)%Mod;
    for (int i=0; i<n; i++){
        for (int j=i+1; j<n; j++){
            while (matrix[j][i]){
                int t=matrix[i][i]/matrix[j][i];
                for (int k=i; k<n; k++){
                    matrix[i][k]=(matrix[i][k]+(Mod-matrix[j][k])*t)%Mod;
                    swap(matrix[j][k],matrix[i][k]);
                }
                ret=Mod-ret;
            }
        }
        if (!matrix[i][i]) return 0;
        ret=ret*matrix[i][i]%Mod;
    }
    return ret;
}
\end{verbatim}

~\\
~\\
\subsubsection{高斯消元}
\begin{verbatim}
/*
 * 把矩阵消成上三角矩阵,行列式就是对角线相乘
 * 适用于模数为质数的情况
 * 使用时注意范围溢出问题
 * 时间复杂度O(n^3)
*/

void extended_gcd(int a, int b, int &x, int &y){
    if (!b){
        x=1,y=0;
        return;
    }
    extended_gcd(b,a%b,y,x);
    y-=(a/b)*x;
}

int gauss(int matrix[][kMaxN], int n){
    long long ret=1,inv=1;
    for (int i=0; i<n; i++)
        for (int j=0; j<n; j++)
            matrix[i][j]=(matrix[i][j]%Mod+Mod)%Mod;

    for (int i=0; i<n; i++){
        int absMax=matrix[i][i],pos=i;
        for (int j=i+1; j<n; j++){
            if (abs(matrix[j][i])>abs(absMax)){
                pos=j;
                absMax=matrix[j][i];
            }
        }
        if (i!=pos){
            for (int j=0; j<n; j++){
                swap(matrix[i][j],matrix[pos][j]);
            }
            ret=Mod-ret;
        }
        if (!matrix[i][i]) return 0;
        
        for (int k=i+1; k<n; k++){
            if (matrix[k][i]==0) continue;
            int d=gcd(abs(matrix[i][i]),abs(matrix[k][i]));
            int lcm=abs(matrix[i][i]*matrix[k][i])/d,
				mul1=lcm/matrix[i][i],mul2=lcm/matrix[k][i];
            inv=inv*mul2%Mod;
            if (inv<0) inv+=Mod;
            for (int j=i; j<n; j++){
                matrix[k][j]=matrix[k][j]*mul2-matrix[i][j]*mul1;
                matrix[k][j]%=Mod;
                if (matrix[k][j]<0) matrix[k][j]+=Mod;
            }
        }
        ret=ret*matrix[i][i]%Mod;
    }

    int x,y;
    extended_gcd(inv,Mod,x,y);
    x=(x%Mod+Mod)%Mod;
    ret=ret*x%Mod;
    return ret;
}

\end{verbatim}
