\subsection{左偏树}
\begin{verbatim}
/*
 * 左偏树应用快速合并两个堆
 * 左偏树的节点有4个信息,分别是key,到最近的外部节点的距离dist
 * 左右子树的指针left,right
 *
 * 左偏树的性质:
 * 1、节点的key小于等于左右子节点的key
 * 2、节点的左子树的dist不小于右子树的dist
 * 3、节点的dist等于其右子树的dist+1,特别地空子树的dist=-1
 * 4、一棵N个节点的左偏树的距离最多为log(N+1)-1
 *
 * 左偏树的操作是建立在merge之上,合并两棵左偏树A,B,以最小堆为例:
 * 1、若A为空树,则返回B,若B为空树,返回A
 * 2、如果A的key大于B的key,则交换A,B,然后继续合并
 * 3、将B和A的右子树合并,如果合并之后A的右子树的dist大于左子树的
 * dist, 则交换A的左右子树。若合并后A的右子树为空,额A的dist为0,
 * 否则A的dist等于右子树的dist+1
 * 
 * 左偏树的操作:
 * 1、插入:相当于合并两棵左偏树
 * 2、删除根节点:合并根节点的左又子树
 * 3、查询最值:根节点
 *
 * 给出N个节点,构建一棵N个节点的左偏树。
 * 1、首先把所有节点加入到队列里面,每一个节点都可以看作一棵左偏树
 * 2、每次选队列最前端的两个节点合并,把合并后的左偏树放到队尾。
 * 3、这样合并N-1次就可以得到一棵左偏树,时间复杂度为O(N)
 * 
 * 用指针实现的左偏树,在实现清除操作时,要注意指针所指的数据不能被
 * delete多次,否则会出现运行错误
 *
 * 左偏树的操作的时间复杂度:
 * 构建        :O(N)
 * 插入        :O(logN)
 * 查询最值       :O(1)
 * 合并        :O(logN)
*/

//以下是用指针实现的根的key为最小值的左偏树
#include <cstdio>
#include <cstring>
#include <algorithm>
using namespace std;

struct leftistNode{
    int key,dist;
    leftistNode *left,*right;
    leftistNode(int key=0):key(key){
        dist=0;
        left=right=NULL;
    }
};

leftistNode *mergeNode(leftistNode *a, leftistNode *b){
    if (a==NULL) return b;
    if (b==NULL) return a;
    if (a->key>b->key) swap(a,b);
    a->right=mergeNode(a->right,b);
    if (a->right!=NULL){
        if (a->left==NULL || a->left->dist<a->right->dist)
            swap(a->left,a->right);
    }
    if (a->right==NULL){
        a->dist=0;
    } else{
        a->dist=a->right->dist+1;
    }
    return a;
}

struct leftistTree{
    leftistNode *root;

	leftistTree():root(NULL){}

    void insert(leftistNode *rhs){
        root=mergeNode(root,rhs);
    }

    void insert(int key){
        leftistNode *temp=new leftistNode(key);
        insert(temp);
    }

    void del(){
        if (root==NULL) return;
        leftistNode *temp=mergeNode(root->left,root->right);
        delete root;
        root=temp;
    }

    void _clear(leftistNode *root){
        if (root==NULL) return;
        if (root->left!=NULL) _clear(root->left);
        if (root->right!=NULL) _clear(root->right);
        delete root;
    }

    void clear(){
        _clear(root);
        root=NULL;
    }

    leftistNode *askMax(){
        return root;
    }
};

leftistTree merge(leftistTree a, leftistTree b){
    leftistTree ret;
    ret.root=mergeNode(a.root,b.root);
    return ret;
}

/*
 * 非指针版的实现就是事先开一个buffer存储数据
 * 以下是zoj2334的程序
*/

#include <cstdio>
#include <cstring>
#include <algorithm>
using namespace std;

const int kMaxN=100001;

struct leftistNode{
    int key,dist,left,right;

    leftistNode(int key=0):key(key){
        dist=0;
        left=right=0;
    }
};

leftistNode buffer[kMaxN];

int mergeNode(int A, int B){
    if (!A) return B;
    if (!B) return A;
    if (buffer[A].key<buffer[B].key) swap(A,B);
    buffer[A].right=mergeNode(buffer[A].right,B);
    if (!buffer[B].left || 
        buffer[buffer[A].left].dist<buffer[buffer[A].right].dist)
            swap(buffer[A].left,buffer[A].right);
    if (!buffer[A].right){
        buffer[A].dist=0;
    } else buffer[A].dist=buffer[B].dist+1;
    return A;
}

struct leftistTree{
    int root;
    leftistTree(int root=0):root(root){}

    void insert(int pos){
        root=mergeNode(root,pos);
    }

    int del(){
        int temp=root;
        root=mergeNode(buffer[root].left,buffer[root].right);
        return temp;
    }

    int top(){
        return root;
    }
};

leftistTree merge(leftistTree &a, leftistTree &b){
    return leftistTree(mergeNode(a.root,b.root));
}

int n;
int number[kMaxN],father[kMaxN];
leftistTree heap[kMaxN];

void init(){
    for (int i=1; i<=n; i++) scanf("%d",&number[i]);
}

int find(int k){
    int x,y=k;
    while (father[y]!=y) y=father[y];
    while (k!=y){
        x=father[k];
        father[k]=y;
        k=x;
    }
    return y;
}

void solve(){
    int x,y,m,total=0;
    for (int i=1; i<=n; i++){
        father[i]=i;
        buffer[++total]=leftistNode(number[i]);
        heap[i].root=i;
    }
    scanf("%d",&m);
    for (int i=0; i<m; i++){
        scanf("%d%d",&x,&y);
        x=find(x),y=find(y);
        if (x==y){
            printf("%d\n",-1);
        } else{
            father[y]=x;
            heap[x]=merge(heap[x],heap[y]);
            int pos=heap[x].del();
            buffer[pos]=leftistNode(buffer[pos].key/2);
            printf("%d\n",buffer[pos].key);
            heap[x].insert(pos);
        }
    }
}

int main(){
    while (scanf("%d",&n)!=EOF){
        init();
        solve();
    }
    return 0;
}


\end{verbatim}

