\subsection{计算几何基础}
\subsubsection{基本定理}
1、$\triangle ABC$的外接圆半径$R=abc/4S$,其中$a,b,c,S$分别表示三角形的三条边的边长以及三角形的面积。这个等式可以用正弦定理证明。\par

2、圆内接四边形的对角互补,设四条边的长度分别为$a,b,c,d$,半周长$p=(a+b+c+d)/2$,则内接四边形的面积$S=\sqrt{(p-a)(p-b)(p-c)(p-d)}$。其外接圆半径$R=\frac{\sqrt{(ac+bd)(ad+bc)(ab+cd)}}{4S}$。在等边长的四边形中,圆内接四边形的面积最大。\par

3、所有的正多边形都有外接圆,外接圆的圆心和正多边形的中心重合。变长为$a$的正$n$边形外接圆的半径$R_n=\frac{a}{2\mathbf{sin}(\frac{\pi}{n})}$,面积$S=\pi R_n^2=\frac{\pi a^2}{4\mathbf{sin}(\frac{\pi}{n})^2}$。 \par

\subsubsection{坐标旋转}
将$(x_1,y_1)$旋转$\beta$,则得到的坐标$(x_2,y_2)$为$(x_1*cos\beta-y_1*sin\beta,x_1*sin\beta+y_1*cos\beta)$

~\\

\subsubsection{pick定理}
(整数格子点简单多边形的面积、内部点数、边界点数关系):\par
设F为平面上以格子点为定点的单纯多边形,则其面积为:S=b/2+i-1。 \par
b为多边形边上点格点的个数,i为多边形内部格点的个数。 \par
可用其计算多边形的面积,边界格点数或内部格点数。 \par

~\\

\subsubsection{点到线段之间的距离}
首先利用三角形底边与面积关系求出高(点到直线距离),然后用点乘求出向量$\overrightarrow{AB}$与向量$\overrightarrow{BC}$的
夹角,如果为钝角,则高在三角形内,否则在三角形外,如果在三角形外的话,那么点到线段的距离必然是到两端点的距离最短者。\par

\begin{verbatim}
double PointLineDist(point C, point A, point B){
        double ret=(B-A)*(C-A)/sqrt((B-A)^(B-A));
        if (((C-B)^(B-A))>0) return sqrt((B-C)^(B-C));
        if (((C-A)^(A-B))>0) return sqrt((A-C)^(A-C));
        return abs(ret);
}
\end{verbatim}

\subsubsection{线段和直线方程}
\begin{verbatim}
struct point{
    double x,y;
};

point operator -(const point &a, const point &b){
    point ret;
    ret.x=a.x-b.x; ret.y=a.y-b.y;
    return ret;
}

/* 叉乘 A*B=|A||B|cos(<A,B>)若A叉乘B<0则A在B的左手边 */
double operator *(const point &a, const point &b){
    return a.x*b.y-a.y*b.x;
}

/* 点乘 A^B=|A||B|sin(<A,B>) */
double operator ^(const point &a, const point &b){
    return a.x*b.x+a.y*b.y;
}

/* 获取线段ab的中点 */
point midpoint(point &a, point &b){
    point ret;
    ret.x=(a.x+b.x)/2; ret.y=(a.y+b.y)/2;
    return ret;
}

/* 计算经过点a和点b的直线的方程Ax+By=C */
void getABC(double &A, double &B, double &C, point &a, point &b){
    A=b.y-a.y;
    B=a.x-b.x;
    C=A*a.x+B*a.y;
}

/* 计算线段ab的中垂线,直线ab的方程为Ax+By=C */
void getmidline(double &A, double &B, double &C, point &a, point &b){
    swap(A,B);
    A=-A;
    point p=midpoint(a,b);
    C=A*p.x+B*p.y;
}

/* 计算直线A1x+B1y=C1与直线A2x+B2y=C2的交点 */
point getIntersection(double A1, double B1, double C1,
                      double A2, double B2, double C2){
    double det=A2*B1-A1*B2;
    if (abs(det)<eps){
        /* 两直线平行 */
    }

    point ret;
    ret.x=(B1*C2-B2*C1)/det;
    ret.y=(A2*C1-A1*C2)/det;
    return ret;
}
\end{verbatim}


\subsubsection{两圆交点} 
\begin{verbatim}
#include <cmath>
#include <cstdio>
#include <iostream>
using namespace std;

const double eps=1e-8;

struct point{
    double x,y;
};

struct Circle{
    point C;
    double R;
};

double sqr(double x){
    return x*x;
}

double Distance2(point &a, point &b){
    return sqr(a.x-b.x)+sqr(a.y-b.y);
}

void Circle_Intersection(Circle A, Circle B){
    double Dist=sqrt(Distance2(A.C,B.C));
    if (Dist-A.R-B.R>eps || fabs(A.R-B.R)-Dist>eps){
        //no Intersection
        cout<<"No Intersection"<<endl;
        return;
    }
    double a=2*A.R*(A.C.x-B.C.x);
    double b=2*A.R*(A.C.y-B.C.y);
    double c=B.R*B.R-A.R*A.R-Distance2(A.C,B.C);
    double p=a*a+b*b;
    double q=-2*a*c;
    double r=c*c-b*b;
    double cos1=(-q+sqrt(q*q-4*p*r))/(2*p);
    double cos2=(-q-sqrt(q*q-4*p*r))/(2*p);
    double sin1=sqrt(1-cos1*cos1);
    double sin2=sqrt(1-cos2*cos2);

    point ret1,ret2;
    ret1.x=cos1*A.R+A.C.x; ret1.y=sin1*A.R+A.C.y;
    ret2.x=cos2*A.R+A.C.x; ret2.y=sin2*A.R+A.C.y;
    if (fabs(Distance2(ret1,B.C)-B.R*B.R)>eps)
        ret1.y=-sin1*A.R+A.C.y;
    if (fabs(Distance2(ret2,B.C)-B.R*B.R)>eps)
        ret2.y=-sin2*A.R+A.C.y;
    //the same point
    if (fabs(Distance2(ret1,ret2))<eps){
        ret1=ret2;
    }
}
\end{verbatim}


\subsection{圆与简单多边形面积交}
\begin{verbatim}
#include <cmath>
#include <cstdio>
#include <cstdlib>
#include <cstring>
#include <map>
#include <set>
#include <queue>
#include <stack>
#include <vector>
#include <sstream>
#include <iostream>
#include <algorithm>
using namespace std;

const int MAXN=128;
const double eps=1e-8;
const double pi=acos(-1.0);

struct point{
    double x,y;
    point(){
        x=0;
        y=0;
    }
    point(double x,double y):x(x),y(y){}
    bool operator ==(point b)const{
        return x==b.x && y==b.y;
    }
};

struct polygon{
    int n;
    point points[MAXN];
    int size(){
        return n;
    }
    point& operator [](int i){
        return points[i];
    }
    void resize(int x){
        n=x;
    }
};

inline double length(point a){
    return sqrt(a.x*a.x+a.y*a.y);
}

inline point operator +(point a,point b){
    return point(a.x+b.x,a.y+b.y);
}

inline point operator -(point a,point b){
    return point(a.x-b.x,a.y-b.y);
}

inline point operator *(point a,double t){
    return point(a.x*t,a.y*t);
}

inline point operator /(point a,double t){
    return point(a.x/t,a.y/t);
}

inline double operator *(point a,point b){
    return a.x*b.x+a.y*b.y;
}

inline double operator ^(point a,point b){
    return a.x*b.y-a.y*b.x;
}

inline double angle(point a,point b){
    double ans=fabs(atan2(a.y,a.x)-atan2(b.y,b.x));
    if (ans>pi) ans=pi*2-ans;
    return ans;
}

const point no_solution(0,0);

point intersection_to_circle(point p,point q,double r){
    point u=q-p;
    double A=u*u;
    double B=2*(p*u);
    double C=p*p-r*r;
    double delta=B*B-4*A*C;
    if (delta<-eps) return no_solution;
    else if (fabs(delta)<eps){
        double t=-B/(2*A);
        if (t<-eps || t>1.0+eps)
            return no_solution;
        return p+u*t;
    } else{
        double t1=(-B-sqrt(delta))/(2.0*A);
        double t2=(-B+sqrt(delta))/(2.0*A);
        double t;
        bool flag1=(t1>-eps && t1<1.0+eps);
        bool flag2=(t2>-eps && t2<1.0+eps);
        if (!flag1 && !flag2) return no_solution;
        else if (!flag1) t=t2;
        else t=t1;
        return p+u*t;
    }
}

point gravity_center(polygon p){
    point ans(0,0);
    double area=0;
    for (int i=0;i<p.size();i++){
        double now_area=p[i]^p[i+1];
        point now_center=(p[i]+p[i+1])/3.0;
        area+=now_area;
        ans=ans+now_center*now_area;
    }
    ans=ans/area;
    return ans;
}

double intersection_area(polygon p,point c,double r){
    double ans=0;
    for (int i=0;i<p.size();i++){
        point a=p[i]-c;
        point b=p[i+1]-c;
        double la=length(a);
        double lb=length(b);
        double now_area;
        int now_sign;
        if ((a^b)>0) now_sign=1;
        else now_sign=-1;
        double phi=angle(a,b);
        if (la<r+eps && lb<r+eps){
            now_area=fabs(a^b);
        } else if (la<r+eps){
            point c=intersection_to_circle(b,a,r);
            double alpha=angle(a,c);
            double beta=phi-alpha;
            now_area=la*r*sin(alpha)+r*r*beta;
        } else if (lb<r+eps){
            point c=intersection_to_circle(a,b,r);
            double alpha=angle(b,c);
            double beta=phi-alpha;
            now_area=lb*r*sin(alpha)+r*r*beta;
        } else{
            point c=intersection_to_circle(a,b,r);
            point d=intersection_to_circle(b,a,r);
            if (c==no_solution) now_area=r*r*phi;
            else{
                double alpha=angle(c,d);
                double beta=phi-alpha;
                now_area=r*r*sin(alpha)+r*r*beta;
            }
        }
        now_area*=now_sign;
        ans+=now_area;
    }
    return ans*0.5;
}

point center;
polygon p;
double v0,theta,t,g,R;

void init(){
    double x,y;
    cin>>p.n;
    for (int i=0; i<p.n; i++){
        cin>>x>>y;
        p.points[i]=point(x,y);
    }
    p.points[p.n]=p.points[0];
}

void solve(){
    printf("%.2lf\n",intersection_area(p,center,R));
}

int main(){
    while (cin>>center.x>>center.y>>R){
        if (!center.x && !center.y && !v0 && !theta && !t && !g && !R) break;
        init();
        solve();
    }
    return 0;
}
\end{verbatim}


\subsection{最小圆覆盖}
\begin{verbatim}
/*
 * 最小圆覆盖的思想:若第i+1个点不在前i个点的最小圆内,则第i+1个点必然在
 * 前i+1个点的最小圆的边上。
 * 最坏情况下时间复杂度是O(n^3),期望是O(n)
*/

#include <cmath>
#include <cstdio>
#include <iostream>
#include <algorithm>
using namespace std;

const double eps=1e-8;
const int MAXN=1001;

struct point{
    double x,y;

    point(double x=0, double y=0):x(x),y(y){}
    double sqr(double x){ return x*x; }
    double dist2(const point &b){ return sqr(x-b.x)+sqr(y-b.y); }
    double dist(const point &b){ return sqrt(dist2(b)); }
    point operator -(const point &b){ return point(x-b.x,y-b.y); }
    double operator *(const point &b){ return x*b.y-y*b.x; }
};

struct circle{
    point center;
    double R;

    circle(){}
    circle(const point &center, double R):center(center),R(R){}

    bool inside(const point &p){
        return center.dist(p)<R+eps;
    }
};

int n;
point P[MAXN];

void init(){
    double x,y;
    for (int i=0; i<n; i++){
        scanf("%lf%lf",&x,&y);
        P[i]=point(x,y);
    }
    random_shuffle(P,P+n);
}

circle getCircleWith2Point(point &a, point &b){
    return circle(point((a.x+b.x)/2,(a.y+b.y)/2),a.dist(b)/2);
}

//calculate the area of triangle ABC
double Area(point &a, point &b, point &c){
    return fabs((b-a)*(c-a)/2.0);
}

//Rotate vector 1/2BA to vector BR
circle calc(point &a, point &b, double sinA, double cosA, double scale, double R){
    double x=(b.x-a.x)*scale,y=(b.y-a.y)*scale;
    return circle(point(a.x+cosA*x-sinA*y,a.y+cosA*y+sinA*x),R);
}

/*
 * R=abc/4S, where S is the area of triangle ABC.
 * calculate the vector 1/2BA and rotate it to BR.
 * when we know vector BR, we can calculate point R.
*/

circle getCircleWith3Point(point &a, point &b, point &c){
    double r=a.dist(b)*b.dist(c)*c.dist(a)/(4*Area(a,b,c));
    double halfAB=a.dist(b)/2.0;
    double lenBR=sqrt(r*r-halfAB*halfAB);
    double scale=r/(halfAB*2);
    circle Circle=calc(a,b,lenBR/r,halfAB/r,scale,r);
    if (Circle.inside(c)) return Circle;
    return calc(a,b,-lenBR/r,halfAB/r,scale,r);
}

/*
 * we know the minimalCircle of point 1..i, consider the (i+1)th point.
 * if the (i+1)th point is in the minimalCircle of point 1..i, consider (i+2)th point,
 * otherwise, the (i+1)th point is on the edge of the minimalCircle of 1..i+1.
 *
 * so we can add point to the minimalCircle one by one and maintain the minimalCircle.
 * when the point add by random order, the exception complexity of the algorithm is O(n)
*/

void solve(){
    circle Circle=getCircleWith2Point(P[0],P[1]);
    for (int i=2; i<n; i++){
        if (Circle.inside(P[i])) continue;
        Circle=getCircleWith2Point(P[0],P[i]);
        for (int j=0; j<i; j++){
            if (Circle.inside(P[j])) continue;
            Circle=getCircleWith2Point(P[i],P[j]);
            for (int k=0; k<j; k++){
                if (Circle.inside(P[k])) continue;
                Circle=getCircleWith3Point(P[i],P[j],P[k]);
            }
        }
    }
    printf("%.2lf %.2lf %.2lf\n",Circle.center.x,Circle.center.y,Circle.R);
}

int main(){
    while (scanf("%d",&n) && n!=0){
        init();
        solve();
    }
    return 0;
}
\end{verbatim}
