\subsubsection{后缀数组}

\begin{verbatim}
Suffix基本步骤:
1、对字符串出现的字符进行排序,如果字符串出现的不同字母个数相对较少就用基数排序,在字符串的最后加上一个字典序最小的字母,这样可以防止例如AAAAA的时候,Suffix无法计算出Rank。如果遇到多个字符串,一般用不同且未出现过的字符隔开。

2、计算SA、Rank以及height数组。

3、后缀数组最常用到height数组的性质,height[i]表示排名为i和i-1的后缀的最长公共前缀。

Suffix题目类型:
1、求两个串后缀的最长公共前缀:对height做一次RMQ处理,然后就可以在O(1)时间内回答。

2、可重叠最长重复字串:由于可重复,所以直接求height数组中的最大值即可。

3、不可重叠最长重复字串(pku1743):二分字串长度len,把height中连续>=len的分在一个组,然后在这个组找到最大、最小的Sa[i],Sa[j]如果Sa[i]-Sa[j]>=len就有解。

4、可重叠的k次最长重复字串(pku3261):同样二分字串长度len,把height中连续>=len的分在一个组,然后判断这个组元素个数是否不小于k。

5、不同字串的个数(spoj705):每个子串一定是某个后缀的前缀,那么原问题等价于求所有后缀之间的不相同的前缀的个数。如果所有的后缀按照suffix(sa[1]), suffix(sa[2]),...,
suffix(sa[n])的顺序计算,不难发现,对于每一次新加进来的后缀suffix(sa[k]),它将产生n-sa[k]+1个新的前缀。但是其中有height[k]个是和前面的字符串的前缀是相同的。所以suffix(sa[k])将“贡献”出n-sa[k]+1-height[k]个不同的子串。累加后便是原问题的答案。这个做法的时间复杂度为O(n)。

6、一个串的最长回文字串(pku3774):把原串反过来加在原串后,用未出现字符隔开,然后对height做RMQ处理,然后问题就变成求两个串某两个后缀的最长公共字串。

7、两个串的最长连续公共字串:用一个未出现的符号连接两串,然后求满足Sa[i],Sa[i-1]属于不同串的height[i]的最大值。

8、长度不小于k 的公共子串的个数(pku3415):二分答案len,然后把height分组,看是否存在一组height值>=len且出现在每个串。
\end{verbatim}

\begin{verbatim}
/* 
 * sa[i]表示排名为i的后缀在原串的起始位置为sa[i](从1开始计数)
 * rank[i]表示以第i个位置开始的后缀在所有后缀中的排名(从1开始计数)
 * sa[rank[i]]==i、rank[sa[i]]==i
 * height[i]表示排名为i的后缀与排名为i-1的后缀的最长公共前缀
*/

#include <string>
#include <cstring>
#include <iostream>
using namespace std;

const int MAXN=100001;
const int maxChar=256;

int rank[MAXN],rank1[MAXN],rank2[MAXN],sa[MAXN],
    tmpsa[MAXN],height[MAXN],countrank[MAXN];

void suffixArray(string &st){
    int tot=0,len=st.size();
    int number[maxChar];
    bool appear[maxChar];
    memset(appear,0,sizeof(appear));
    memset(number,0,sizeof(number));
    for (int i=0; i<len; i++) appear[st[i]]=true;
    for (int i=0; i<maxChar; i++)
        if (appear[i]) number[i]=++tot;

    memset(countrank,0,sizeof(countrank));
    for (int i=1; i<=len; i++){
        rank[i]=number[st[i-1]];
        ++countrank[rank[i]];
    }
    for (int i=1; i<=maxChar; i++) countrank[i]+=countrank[i-1];
    for (int i=len; i>=1; i--) sa[countrank[rank[i]]--]=i;

    int l=1;
    while (l<=len){
        for (int i=1; i<=len; i++){
            rank1[i]=rank[i];
            if (i+l<=len) rank2[i]=rank[i+l]; else 
                rank2[i]=0;
        }
        memset(countrank,0,sizeof(countrank));
        for (int i=1; i<=len; i++) ++countrank[rank2[i]];
        for (int i=1; i<=len; i++) countrank[i]+=countrank[i-1];
        for (int i=len; i>=1; i--) tmpsa[countrank[rank2[i]]--]=i;

        memset(countrank,0,sizeof(countrank));
        for (int i=1; i<=len; i++) ++countrank[rank1[i]];
        for (int i=1; i<=len; i++) countrank[i]+=countrank[i-1];
        for (int i=len; i>=1; i--) sa[countrank[rank1[tmpsa[i]]]--]=tmpsa[i];

        rank[sa[1]]=1;
        for (int i=2; i<=len; i++){
            rank[sa[i]]=rank[sa[i-1]];
            if (!(rank1[sa[i]]==rank1[sa[i-1]] 
                  && rank2[sa[i]]==rank2[sa[i-1]]))
                ++rank[sa[i]];
        }
        l+=l;
    }

    l=0; height[1]=0;
    for (int i=1; i<=len; i++)
        if (rank[i]>1){
            int j=sa[rank[i]-1];
            while (i+l<=len && j+l<=len && st[i+l-1]==st[j+l-1]) ++l;
            height[rank[i]]=l;
            if (l) --l;
        } else height[rank[i]]=0;
}
\end{verbatim}
