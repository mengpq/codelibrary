\subsection{FFT高精度乘法}
\begin{verbatim}
#include <string>
#include <complex>
#include <cstring>
#include <iostream>
using namespace std;

const double pi=acos(-1.0);

const int MAXN=200001;

int n;
string a,b;
int ret[MAXN];
complex<double> A[MAXN],B[MAXN];

inline int rev(int value , int n){
    int ret=0;
    for (int i=n/2,j=0; i>0; i/=2,j++)
        if (value & i) ret+=1<<j;
    return ret;
}

void Bit_Reverse(complex<double> *A, int n){
    for (int i=0; i<n; i++){
        int j=rev(i,n);
        if (i<j) swap(A[i],A[j]);
    }
}

void FFT(complex<double> *A, int n, int on){
    Bit_Reverse(A,n);
    complex<double> u,t;
    for (int m=2; m<=n; m*=2){
        complex<double>wm(cos(on*2.0*pi/m),sin(on*2.0*pi/m));
        for (int k=0; k<n; k+=m){
            complex<double>w(1,0);
            for (int j=k; j<k+m/2; j++){
                u=A[j];
                t=w*A[j+m/2];
                A[j]=u+t;
                A[j+m/2]=u-t;
                w=w*wm;
            }
        }
    }

    if (on==-1)
        for (int i=0; i<n; i++) A[i].real()/=n;
}

void init_value(complex<double> *A, string &st, int &n){
    reverse(st.begin(),st.end());
    int t=st.size()%2,temp;
    for (int i=0; i<t; i++) st+='0';
    for (int i=0; i<st.size()/2; i++){
        temp=0; t=i*2;
        for (int j=t+2-1; j>=t; j--) temp=temp*10+(st[j]-'0');
        A[i].real()=temp;
        A[i].imag()=0.0;
    }
    for (int i=st.size()/2; i<n; i++)
        A[i].real()=A[i].imag()=0.0;
}

void init(){
    cin>>a>>b;
    n=1;
    int t1=(a.size()-1)/2+1,t2=(b.size()-1)/2+1;
    while (n<t1*2 || n<t2*2) n*=2;
    init_value(A,a,n);
    init_value(B,b,n);
}

void solve(){
    FFT(A,n,1);
    FFT(B,n,1);
    for (int i=0; i<n; i++) A[i]=A[i]*B[i];
    FFT(A,n,-1);
    for (int i=0; i<n; i++) ret[i]=(long long)(A[i].real()+0.5);
    for (int i=0; i<n; i++) cout<<ret[i]<<endl;
    for (int i=0; i<n; i++){
        ret[i+1]+=ret[i]/100;
        ret[i]%=100;
    }
    //while (n>0 && ret[n]==0) --n;
    //printf("%d",ret[n]);
    //for (int i=n-1; i>=0; i--) printf("%.3d",ret[i]);
    //cout<<endl;
}

int main(){
    init();
    solve();
    return 0;
}
\end{verbatim}
